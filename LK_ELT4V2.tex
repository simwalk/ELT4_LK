%
%    Autor: Simon Walker
%			simon.walker@hsr.ch
%	Lizenz: CC BY-NC-SA
%			Für bilder gelten eventuel andere Lizenzen dise sind jeweils dann als Kommentar hinter dem Include angegeben


%	 Dieses LaTeX Dokument benutzt eine Karteikartenklasse welche von Ronny Bergmann <mail@rbergmann.info> entwickelt wurde version 1.8b
%	 Die Dokumentation und die Karteikartenklasse sind unter folgendem Link erhältlich
%    https://github.com/kellertuer/Kartei

%------------------------------------------------------------
% Lernkarten zum Fach ELT4
% Diese Version ist Ergänzend zu dem PreReleas zu sehen.
% Korigierte Karten haben die gleiche Nummer wie im PreRelease
% Neue Karten beginnen mit der Nummerierung ab 101
%------------------------------------------------------------

\documentclass[a7paper,10pt,grid=none% %Grösse der Karteikarten auswählen (A5-A9)
%,toc			%Kommentarlöschen falls eine Zusammenstellung gewünscht wird
%,print			%Kommentar löschen falls eine Ausdruckversion erstellt werden soll
]{kartei}

\usepackage[utf8]{inputenc} %UTF8
\usepackage{hyperref}

\usepackage{amsmath}
\usepackage{amssymb} %Für Real und Complex Symbol
\usepackage{bm}
\usepackage{paralist} % für compactenum and compactitem
\usepackage{xcolor} %Für Farbige Mathesymbole

\usepackage[T1]{fontenc}% wichtig für Trennung von Wörtern mit Umlauten

\usepackage[ngerman]{babel}% deutsche Trennregeln
\usepackage{microtype}% verbesserter Randausgleich

\usepackage{hyphsubst}
\usepackage{lmodern} %Schriftart

\setlength{\parindent}{0em} %Einrücken verhindern


\usepackage{tikz}
\usetikzlibrary{math}
\usepackage{xstring} %If statements
\usepackage{scalefnt}
\usepackage[european]{circuitikz}

%\usepackage{cancel} %\cancel befehl kürzen %Farbe ändern mit \renewcommand\CancelColor{\color{red}}


\newenvironment{tightcenter}{%
	\setlength\topsep{0pt}
	\setlength\parskip{0pt}
	\begin{center}
	}{%
	\end{center}
}

\usepackage{pgfplots}
\usepgfplotslibrary{fillbetween}

\usepgfplotslibrary{smithchart}
\pgfplotsset{compat=1.11}

%\input{tikz/Electrical_Components.tex}
%Autor: Simon Walker
%Version: 1.0
%Datum: 16.12.2019
% "Eigene" Latex befehle


\newcommand\markangle[6][red]{% [color] {X} {origin} {Y} {mark} {radius}
	% filled circle: red by default
	\begin{scope}
		\path[clip] (#2) -- (#3) -- (#4);
		\draw[color=#1,draw=#1,name path=circle] %opacity=0.5
		(#3) circle (#6mm);
	\end{scope}
	% middle calculation
	\path[name path=line one] (#3) -- (#2);
	\path[name path=line two] (#3) -- (#4);
	\path[%
	name intersections={of=line one and circle, by={inter one}},
	name intersections={of=line two and circle, by={inter two}}
	] (inter one) -- (inter two) coordinate[pos=.5] (middle);
	% bissectrice definition
	\path[%
	name path=bissectrice
	] (#3) -- (barycentric cs:#3=-1,middle=1.2);
	% put mark
	\path[
	name intersections={of=bissectrice and circle, by={middleArc}}
	] (#3) -- (middleArc) node[pos=0.7] {#5}; % node[pos=1.3]
}

\newcommand{\HelpCords}[4]{
	\draw [help lines] (#1,#2) grid (#3,#4);
	\foreach \i in {#1,..., #3}
	\node [below] at (\i,#2) {$\i$};
	\foreach \i in {#2,..., #4}
	\node [left] at (#1,\i) {$\i$};
}


%Eingener Befehl: Erzeugt zwei Minipage nebeneinander
\newcommand{\divTwo}[2]{
\begin{minipage}[t]{0.48\textwidth}
	#1	
\end{minipage}
\begin{minipage}[t]{0.48\textwidth}
	#2
\end{minipage}
}

\begin{document}
	\setcardpagelayout
	
	\fach{ELT4}
	%Korrigierte Karten des PreReleas
	\kommentar{Recap ELT3}

\begin{karte}{Was ist eine Reaktanz}
	Die Reaktanz ($X$) ist der Blindwiderstand. Somit ist er der Imaginärteil der Impedanz ($Z$). 
\end{karte}
	\kommentar{Frequenzabhängigkeit}

\setcounter{CardID}{2}
\begin{karte}{Was ist der Verlustfaktor $d$ und was sagt der Winkel $\delta$ aus?}
	Der Verlustfaktor ist der Kehrwert der Güte.
	\begin{equation*}
	d = \frac{1}{Q} = \frac{\operatorname{Re}}{|\operatorname{Im} Z|} = \tan (\delta)
	\end{equation*}
	Der Verlustwinkel $\delta$ wird mehrheitlich dafür eingesetzt, den Anteil der Wirkleistung elektrisch reaktiver Bauteile zu beschreiben und ist wie folgt definiert:
	\begin{equation*}
		\tan(\delta)=\frac{P}{\omega W}=\frac{\operatorname{Re} Z}{|\operatorname{Im} Z|}=\frac{R}{|X|}
		\qquad \delta = \frac{\pi}{2} - |\varphi|
	\end{equation*}
\end{karte}

	\kommentar{Schwingkreise}
\setcounter{CardID}{4}

\begin{karte}{Was ist ein Schwingkreis und welche Betriebsmodus gibt es?}
	Ein Schwingkreis ist:\\
	\begin{itemize}
		\item ein Netzwerk aus Ohmischen, Kapazitiven und Induktiven Bauteile
		\item ein resonanzfähiges Netzwerk
		\item Kann parallel oder seriell geschaltet werden
	\end{itemize}
	Es gibt 2 Betriebsmodus:
	\begin{itemize}
		\item Freie Schwingung 
		\item Erzwungene Schwingung (Durch Quelle angeregt)
	\end{itemize}
\end{karte}
	\kommentar{Reaktanzeintore (RET)}

\begin{karte}{Was ist ein Reaktanzeintor?}
	Ein Reaktanzeintor hat zwei Anschlüsse, besteht nur aus Reaktanzen und hat somit keine Wirkwiderstände.
	\\[10pt]
	\begin{minipage}{0.32\textwidth}
		Beispiel:
		\scalebox{.8}{%Autor: Simon Walker
%Version: 1.0
%Datum: 15.04.2020
%Lizenz: CC BY-NC-SA

\begin{circuitikz}
	%\HelpCords{-2}{2}{5}{-5}
	
	\draw
	%Knoten
	(2.3, 0) coordinate (n1)
	(2.3, -2) coordinate (n2)
	($(n1)+(0.7,0)$) coordinate (n11)
	($(n2)+(0.7,0)$) coordinate (n22)
	
	%Nur zu Hilfszwecken
%	(n1) node[above] {$n1$}
%	(n2) node[above] {$n2$}
%	(n11) node[above] {$n11$}
%	(n22) node[above] {$n22$}
	
	
	(0, 0) to [american inductor, o-] (n1)
	(n1) to[C, *-*] (n2)
	(n1) -- (n11) to[american inductor] (n22) -- (n2)
	(n2) to[short, -o] (0, -2)
	;
\end{circuitikz}}
	\end{minipage}
	\begin{minipage}{0.65\textwidth}
		Es wird zwischen minimalen und nicht minimalen Eintoren unterschieden. Nicht minimale Eintore können auf ein minimales Eintor reduziert werden ohne das die Änderung von aussen bemerkt wird.
	\end{minipage}\\[10pt]
	Da die RET nur aus $L$ und $C$ bestehen hat das RET bei $\omega = 0$ und bei $\omega \rightarrow \infty$ jeweils eine Pol oder eine Nullstelle.
\end{karte}

\begin{karte}{Wie sieht der Reaktanzverlauf des folgenden RET aus und was für ein Typ hat das RET?\\[5pt]
	%Autor: Simon Walker
%Version: 1.0
%Datum: 15.04.2020
%Lizenz: CC BY-NC-SA

\begin{circuitikz}
	%\HelpCords{-2}{2}{5}{-5}
	
	\draw
	%Knoten
	(2.3, 0) coordinate (n1)
	(2.3, -2) coordinate (n2)
	($(n1)+(0.7,0)$) coordinate (n11)
	($(n2)+(0.7,0)$) coordinate (n22)
	
	%Nur zu Hilfszwecken
%	(n1) node[above] {$n1$}
%	(n2) node[above] {$n2$}
%	(n11) node[above] {$n11$}
%	(n22) node[above] {$n22$}
	
	
	(0, 0) to [american inductor, o-] (n1)
	(n1) to[C, *-*] (n2)
	(n1) -- (n11) to[american inductor] (n22) -- (n2)
	(n2) to[short, -o] (0, -2)
	;
\end{circuitikz}}
	\begin{minipage}{0.55\textwidth}
		%Autor: Jürg Rast
%Datum: 04.04.2012
%Lizenz: CC BY-SA
%Grundversion von https://github.com/jrast/ELT4-Notizen/blob/master/tikzPictures/RET/ReaktanzLTyp.tex

%Geändert von: Simon Walker
%Datum: 15.04.2020
%Lizenz: CC BY-NC-SA


\usepgflibrary{shapes.misc}
\begin{tikzpicture}[smooth, xscale=0.5, yscale=0.5]
% Achsen
\draw[->, thick] (-0.2,0) -- +(8.4,0) node[right] {$\omega$}; % Horizontal
\draw[->, thick] (0,-2) -- +(0,4) node[above] {$X(\omega)$}; % Vertikal

% Plots
\draw[color=green!70!black, thick] plot[domain=0:2] (\x,{\x/8+tan(\x*3/4 r)/8}); % Erster Tan
%\draw[color=green!70!black, thick] plot[domain=2.01:4.23] (\x,{tan(\x r)}); % zweiter Tan
\draw[color=green!70!black, thick] plot[domain=2.4:8] (\x,{ (\x/3.2) -3.8/(\x -1.01) }); % letzte kurve

% Poolstellen
\draw[dashed, thick, draw=red] (2.1,-2.1) -- +(0,4.2); % Poolstelle 1
\node[cross out, draw=red, thick] (wr1) at (2.1,0) {};


\draw[dashed] plot[domain=0:8] (\x, {\x/4});
\node (wL) at (6.1,2.4) {$\omega \rightarrow \infty$};

% Nullstellen
\node[rounded rectangle, draw=blue, thick] at(0,0) {};
\node[rounded rectangle, draw=blue, thick] at(4,0) {};
\end{tikzpicture}
	\end{minipage}
	\begin{minipage}{0.4\textwidth}
		Bei diesem Beispiel handelt sich um ein L-Typ
	\end{minipage}

	Als erstes wird das Verhalten des RET bei DC und bei sehr grossen Frequenzen analysiert. Für DC leiten beide Induktivitäten also ist dort eine NS. Für hohe Frequenzen sperrt die Erste Spule also ist dort eine PS.\\
	Da es bei dieser Schaltung um ein minimales RET handelt und aus drei Bauteilen besteht, hat der Reaktanzverlauf drei Pol/Null stellen. Diese treten abwechselnd auf.
\end{karte}

\begin{karte}{Wie sieht der Reaktanzverlauf des folgenden RET aus?\\[5pt]
		%Autor: Simon Walker
%Version: 1.0
%Datum: 15.04.2020
%Lizenz: CC BY-NC-SA

\begin{circuitikz}
	%\HelpCords{-2}{2}{5}{-5}
	
	\draw
	%Knoten
	(3, 0) coordinate (n1)
	(3, -2) coordinate (n2)
	($(n1)+(0.7,0)$) coordinate (n11)
	($(n2)+(0.7,0)$) coordinate (n22)
	(0.7, 0) coordinate (n3)
	(0.7, -2) coordinate (n4)
	(0, 0) coordinate (n5)
	(0, -2) coordinate (n6)
	
	
	%Nur zu Hilfszwecken
%	(n1) node[above] {$n1$}
%	(n2) node[above] {$n2$}
%	(n11) node[above] {$n11$}
%	(n22) node[above] {$n22$}
%	(n3) node[above] {$n3$}
%	(n4) node[above] {$n4$}
%	(n5) node[above] {$n5$}
%	(n6) node[above] {$n6$}
	
	(n5) to[short,o-] (n3)
	(n3) to[C, *-*] (n4)
	(n3) to [american inductor] (n1)
	(n1) to[C, *-*] (n2)
	(n1) -- (n11) to[american inductor] (n22) -- (n2)
	(n2) to[short, -o] (n6)
	;
\end{circuitikz}}
	\centering
	%Autor: Jürg Rast
%Datum: 04.04.2012
%Lizenz: CC BY-SA
%Grundversion von https://github.com/jrast/ELT4-Notizen/blob/master/tikzPictures/RET/ReaktanzPTyp.tex

%Geändert von: Simon Walker
%Datum: 15.04.2020
%Lizenz: CC BY-NC-SA

\usepgflibrary{shapes.misc}
\begin{tikzpicture}[smooth, xscale=0.7, yscale=0.5]
% Achsen
\draw[->, thick] (-0.2,0) -- +(6.2,0) node[right] {$\omega$}; % Horizontal
\draw[->, thick] (0,-2) -- +(0,4) node[above] {$X(\omega)$}; % Vertikal

% Plots
\draw[color=green!70!black, thick] plot[domain=0.01:0.9] (\x, {tan( (\x *1.2) r )        }); % Erster Tan
\draw[color=green!70!black, thick] plot[domain=1.3:3.15] (\x,    {tan( (\x*1.2 -2.7) r)    }); % Erster Tan
\draw[color=green!70!black, thick] plot[domain=6.5:8.6] (\x-2.8,  { ((0.25 * \x) -1/(\x -6)) -2 }); % letzte kurve

% Poolstellen
\draw[dashed, thick, draw=red] (1,-2.1) -- +(0,4.2); % Poolstelle 1
\node[cross out, draw=red, thick] at (1,0) {};

\draw[dashed, thick, draw=red] (3.4,-2.1) -- +(0,4.2); % Poolstelle 2
\node[cross out, draw=red, thick] at (3.4,0) {};



% Nullstellen
\node[rounded rectangle, draw=blue, thick] at(0,0) {};
\node[rounded rectangle, draw=blue, thick] at(2.25,0) {};

\end{tikzpicture}\\
	\vspace{-5pt}
	\flushleft
	Als erstes wird das Verhalten des RET bei DC und bei sehr grossen Frequenzen analysiert. Für DC leiten beide Induktivitäten also ist dort eine NS. Für hohe Frequenzen leiten Beide Kapazitäten also ist dort ebenfalls NS.\\
	Da es bei dieser Schaltung um ein minimales RET handelt und aus vier Bauteilen besteht, hat der Reaktanzverlauf vier Pol/Null stellen. Diese treten abwechselnd auf.
\end{karte}

\begin{karte}{Ist folgendes RET ein minimales Eintor?\\[5pt]
		\input{tikz/RET3.tex}}
	Nein! Denn es hat ein Ring aus Induktivitäten. In diesem Ring kann ein Strom fliesen. Dieser Strom ist von aussen nicht sichtbar. Eine Induktivität kann ersatzlos gestrichen werden, wenn dementsprechend die anderen Induktivitäten angepasst werden. Danach ist von aussen kein Unterschied feststellbar.\\
	\begin{center}
		\input{tikz/RET3Ring.tex}
	\end{center}
\end{karte}

\begin{karte}{Welche Eigenschaften eines RET sind in der Praxis am unwahrscheinlichsten?}
	Es gibt zwei Probleme:
	\begin{compactitem}
		\item Impedanz wird nie 0 erreichen
		\item Die Impedanz kann nicht undendlich gross werden.
	\end{compactitem}
	Allerdings wird die Impedanz sich nicht komplett anders verhalten als die Reaktanz.
	%Autor: Simon Walker
%Version: 1.0
%Datum: 18.04.2020
%Lizenz: CC BY-NC-SA

\begin{tikzpicture}[smooth, xscale=0.7, yscale=0.7]
% Achsen
\draw[->, thick] (-0.2,0) -- +(5.2,0) node[right] {$\omega$}; % Horizontal
\draw[->, thick] (0,-2) -- +(0,4) node[above] {$\color{red}X \color{black}, \color{green} \left|Z\right|$}; % Vertikal

% Plots
\draw[color=red, thick] plot[domain=0.01:0.9] (\x*1.5,{tan((\x *1.2) r )}); % Erster Tan
\draw[color=red, thick] plot[domain=1.3:3.15] (\x*1.5,{tan((\x*1.2 -2.7) r)}); % Zweiter Tan
%Impedanz
%\draw[color=green, thick] plot[domain=0.01:0.9] (\x*1.5,{0.3+0.6*tan((\x *1.2) r )}); % Erster Tan

\draw [green, thick] plot [smooth] coordinates{
(0, 0.3) (0.5, 0.5) (1, 1) 
(1.5, 1.8)  (2, 1) (2.5, 0.5)
(3, 0.3) (3.5, 0.3) (4, 0.5) 
(4.5, 1) (5, 1.6) (5.5, 1.8)};


\draw [blue, thick] (3.34, 0) circle[radius=0.25];
\draw [blue, thick] (0, 0) circle[radius=0.25];
\draw [brown, thick] (1.35, 1.85) circle[radius=0.25];
\draw [brown, thick] (1.95, -2.15) circle[radius=0.25];
\draw [brown, thick] (4.73, 1.9) circle[radius=0.25];

\node [below] at (3.8, -0.2) {$\scriptstyle \color{blue} \left|Z\right| \ne 0 \Omega$};
\node at (3, 2) {$\scriptstyle \color{brown} \left|Z\right| \not\rightarrow \infty$};

\end{tikzpicture}

\end{karte}



	\kommentar{Leitungen}

\setcounter{CardID}{25}
\begin{karte}{Weshalb und wann entstehen Reflexionen?}
	Reflexionen entstehen immer dann wenn eine Leitung nicht optimal abgeschlossen wird $Z_A \ne Z_0$. Je Grösser die Differenz zwischen $Z_A$ und $Z_0$, desto mehr wird Reflektiert.\\
	Das Verhältnis zwischen der Leistung welche reflektiert wird $P_{-}$ mit der Leistung welche in den Abschlusswiderstand eindringt $P_{+}$ ist über den Reflexionsfaktor definiert:
	\begin{equation*}
		\frac{P_{-}}{P_{+}}=|\Gamma|^{2}
	\end{equation*}
\end{karte}

	\kommentar{Zweitore}

\setcounter{CardID}{37}
\begin{karte}{Was versteht man im Bezug auf Zweitore unter Passivität, Reziprozität und Symmetrie?}
	\textbf{Passivität:} Die Leistung die das Zweitor aufnimmt ist grösser als 0 ($P\ge 0$)\\
	\textbf{Reziprozität:} Ist die Symmetrie bezüglich dem Ausgang. Die Ströme durch die Amperemeter (blau umkreist) müssen gleich sein.\\
	\textbf{Symmetrie:} Die Ströme durch die Quelle (grün umkreist) müssen ebenfalls gleich sein.
	
	\centering{\scalebox{.7}{%Autor: Simon Walker
%Version: 1.0
%Datum: 25.06.2020
%Lizenz: CC BY-NC-SA


\begin{circuitikz}
%Symbol1
\draw[thick] (-1, -0.8) rectangle (1, 0.8);
\draw (-1, 0.5) to[short, -o] (-2, 0.5);
%\draw[thick] (-1,0.5) -- (-2,0.5);
%\draw[thick, fill=white] (-2,0.5) circle (0.1);
\draw (-1, -0.5) to[short, -o] (-2, -0.5);
%\draw[thick] (-1,-0.5) -- (-2,-0.5);
%\draw[thick, fill=white] (-2,-0.5) circle (0.1);

\draw[thick] (1,0.5) -- (2,0.5);
\draw[thick, fill=white] (2,0.5) circle (0.1);
\draw[thick] (1,-0.5) -- (2,-0.5);
\draw[thick, fill=white] (2,-0.5) circle (0.1);

%Strompfeile
\draw[red, thick] (-1.5, 0.5) node[above, red] {$i_1$} ++(-0.15, 0.15) -- ++(0.15, -0.15) -- ++(-0.15, -0.15);

\draw[red, thick] (1.5, 0.5) node[above, red] {$i_2$} ++(0.15, 0.15) -- ++(-0.15, -0.15) -- ++(0.15, -0.15);

%Quelle
\draw (-2,0.5) to[short] (-2,1) to[short] (-2.5, 1)	to[american voltage source] (-2.5 ,-1) to[short] (-2,-1) to[short] (-2, -0.5);

%Amperemeter
\draw (2,0.5) to[short] (2,1) to[short] (2.5, 1) to node[draw,circle,fill=white] {A} (2.5 ,-1) to[short] (2,-1) to[short] (2, -0.5);

%Einkreisen
\draw[green, very thick] (-1.5, 0.5) circle (0.5);
\draw[blue, very thick] (1.5, 0.5) circle (0.5);


\begin{scope}[shift={(7,0)}]
	%Symbol2
	\draw[thick] (-1, -0.8) rectangle (1, 0.8);
	\draw (-1, 0.5) to[short, -o] (-2, 0.5);
	%\draw[thick] (-1,0.5) -- (-2,0.5);
	%\draw[thick, fill=white] (-2,0.5) circle (0.1);
	\draw (-1, -0.5) to[short, -o] (-2, -0.5);
	%\draw[thick] (-1,-0.5) -- (-2,-0.5);
	%\draw[thick, fill=white] (-2,-0.5) circle (0.1);
	
	\draw[thick] (1,0.5) -- (2,0.5);
	\draw[thick, fill=white] (2,0.5) circle (0.1);
	\draw[thick] (1,-0.5) -- (2,-0.5);
	\draw[thick, fill=white] (2,-0.5) circle (0.1);
	
	%Strompfeile
	\draw[red, thick] (-1.5, 0.5) node[above, red] {$i_1$} ++(-0.15, 0.15) -- ++(0.15, -0.15) -- ++(-0.15, -0.15);
	
	\draw[red, thick] (1.5, 0.5) node[above, red] {$i_2$} ++(0.15, 0.15) -- ++(-0.15, -0.15) -- ++(0.15, -0.15);
	
	%Quelle
	\draw (-2,0.5) to[short] (-2,1) to[short] (-2.5, 1)	to node[draw,circle,fill=white] {A} (-2.5 ,-1) to[short] (-2,-1) to[short] (-2, -0.5);
	
	%Amperemeter
	\draw (2,0.5) to[short] (2,1) to[short] (2.5, 1) to[american voltage source] (2.5 ,-1) to[short] (2,-1) to[short] (2, -0.5);
	
	%Einkreisen
	\draw[green, very thick] (1.5, 0.5) circle (0.5);
	\draw[blue, very thick] (-1.5, 0.5) circle (0.5);
	
\end{scope}


\end{circuitikz}
}}
\end{karte}

	\kommentar{EMV}

\setcounter{CardID}{41}
\begin{karte}{Welche Kopplungsarten von Störungen werden unterschieden?}
	Die Kopplungsarten lassen sich in zwei Grundtypen unterscheiden (Nahfeld- und Fernfeld-Effekte)\\[4pt]
	\textbf{Nahfeld-Effekte:}
	\begin{compactitem}
		\item Kapazitiver Kopplung via elektrischem Feld bzw. von Flächen/Leitern unterschiedlichen Potentials
		\item Induktiver Kopplung via Magnetfeld bzw. von Strömen
		\item Galvanischer Kopplung über gemeinsame Impedanzen (z.B. stromführende Leiterabschnitte)
	\end{compactitem}
	\vspace{4pt}
	\textbf{Fernfeld-Effekte:}
	\begin{compactitem}
		\item Kopplung über Wellenausbreitung auf Leitungen
		\item Strahlungskopplung über Wellenausbreitung im freien Raum
	\end{compactitem}
\end{karte}

	
	%Zusätzliche Karten Version 2
	\setcounter{CardID}{100}
	\kommentar{Recap ELT3}
\begin{karte}{Beschreibe die Begriffe im Impedanz Dreieck!}
	Die Begriffe im Impedanz Dreieck lauten wie folgt:\\
	\centering
	\input{tikz/ImpDreieck.tex}
\end{karte}

\begin{karte}{Beschreibe die Begriffe im Admittanz Dreieck!}
	Die Begriffe im Impedanz Dreieck lauten wie folgt:\\
	\centering
	%Autor: Simon Walker
%Version: 1.0
%Datum: 15.07.2020
%Lizenz: CC BY-NC-SA

\begin{tikzpicture}
	%\tikzset{arrow/.style={draw, ->, shorten >=3pt, shorten <=3pt}}
	
	\coordinate (n1) at (0, 0);
	\coordinate (n2) at (3, 0);
	\coordinate (n3) at (3, 2);
	
	\draw (n1) -- (n2) -- (n3) -- cycle;
	
	\draw[->, thick, blue] (n1)++(0, -0.2) -- node[below] {$G$: Konduktanz} ++(n2);
	\draw[->, thick, blue] (n1)++(-0.05, 0.17) -- node[above, sloped] {$Y$: Admittanz} ++(n3);
	\draw[->, thick, blue] (n2)++(0.2, 0) -- node[below, sloped] {$B$: Suszeptanz} ($(n3)+(0.2, 0)$);
	
\end{tikzpicture}
\end{karte}

	\kommentar{Frequenzabhängigkeit}

\begin{karte}{Was ist die Güte (einer Reaktanz) und wie ist sie definiert?}
	Die Güte eines reaktiven Elements ist definiert als das Verhältnis seiner Reaktanz (Imaginärteil der Impedanz) und seiner Resistanz (Realteil der Impedanz):\\
	\begin{equation*}
		Q=\frac{|\operatorname{Im} Z|}{\operatorname{Re} Z}=\frac{|X|}{R}
	\end{equation*}
	Grundsätzlich ist ein Bauteil besser je grösser die Güte ist.
\end{karte}

\begin{karte}{Was ist der Verlustfaktor $d$ und was sagt der Winkel $\delta$ aus?}
	Der Verlustfaktor ist der Kehrwert der Güte.
	\begin{equation*}
	d = \frac{1}{Q} = \frac{\operatorname{Re}}{|\operatorname{Im} Z|} = \tan (\delta)
	\end{equation*}
	Der Verlustwinkel $\delta$ wird mehrheitlich dafür eingesetzt, den Anteil der Wirkleistung elektrisch reaktiver Bauteile zu beschreiben und ist wie folgt definiert:
	\begin{equation*}
	\tan(\delta)=\frac{P}{\omega W}=\frac{\operatorname{Re} Z}{|\operatorname{Im} Z|}=\frac{R}{|X|}
	\qquad \delta = \frac{\pi}{2} - |\varphi|
	\end{equation*}
\end{karte}

\begin{karte}{Was ist die Güte einer Induktivität und einer Kapazität?}
	Für Induktivitäten:
	\begin{equation*}
		Q_L = \frac{X_L}{R_L} = \frac{\omega_L}{R_L} = \omega \tau_L
	\end{equation*}
	Für Kapazitäten:
	\begin{equation*}
		Q_C = \frac{|X_C|}{R_C} = \frac{1}{\omega R_C C} = \frac{1}{\omega \tau_C}
	\end{equation*}
	%TODO mit Theorie ergänzen ev. eine Grafik
\end{karte}
	\kommentar{Schwingkreise}

\begin{karte}{Was ist ein Schwingkreis und welche Betriebsmodus gibt es?}
	Ein Schwingkreis ist:
	\begin{itemize}
		\item ein Netzwerk aus Ohmischen, Kapazitiven und Induktiven Bauteile
		\item ein resonanzfähiges Netzwerk
		\item Kann parallel oder seriell geschaltet werden
	\end{itemize}
	Es gibt 2 Betriebsmodus:
	\begin{itemize}
		\item Freie Schwingung 
		\item Erzwungene Schwingung (Durch Quelle angeregt)
	\end{itemize}
\end{karte}

\begin{karte}{Welche Frequenzen gibt es in einem Schwingkreis?}
	Es gibt drei verschiedene Frequenzen in einem Schwingkreis:
	\begin{itemize}
		\item Eigenfrequenz $\omega_0$\\
		(Resonanzfrequenz des verlustlosen Schwingkreis)
		\item Resonanzfrequenz $\omega_r$\\
		($Im(Z) = 0$, Reine Reelle Impedanzen und Admitanzen)
		\item Extremalfrequenz $\omega_m$\\
		(Frequenz bei welcher die Impedanz maximal oder minimal ist)
	\end{itemize}
\end{karte}

\begin{karte}{Wie ist die Bandbreite eines Schwingkreises definiert?}
	\textbf{(absolute) Bandbreite}:
	\begin{equation*}
		B = f^+ - f^- = \dfrac{\omega^+ - \omega^-}{2 \pi}
	\end{equation*}
	Die Bandbreite ist der Frequenzbereich, in welchem die Impedanz bzw. Admittanz des Resonators sich nur um den Faktor $\sqrt{2}$ vom nächsten (lokalen) Minimum $\omega_m$ unterscheidet.\\
	Oder Formal ausgedrückt: \quad $|Z (\omega^{\pm})|=\sqrt{2}\cdot|Z(\omega_{m})|$\\
	\textbf{relative Bandbreite}:
	\begin{equation*}
	B_{rel} = \frac{f^+ - f^-}{f_m} = \frac{\omega^+ - \omega^-}{\omega_m} =\frac{1}{Q_s}
	\end{equation*}
	Wobei $Q_s$ die Schwingkreisgüte ist.
\end{karte}

\begin{karte}{Was bedeutet Resonanz?}
	\begin{itemize}
		\item Aus der Akustik:\\
		Verstärktes Mitschwingen eines schwingfähigen Systems
		\item Die Resonanzfrequenz $\omega_r$ ist die Frequenz bei welcher die elektrische und magnetische Energie gleich gross sind $W_C = W_L$.
	\end{itemize}
	Bei der Resonanzfrequenz gilt folglich:
	\begin{itemize}
		\item $|X_C| = |X_L|$
		\item $Im(Z) = 0$
		\item $\tan (\varphi) = 0 \quad \rightarrow \quad \varphi = 0$
	\end{itemize}
\end{karte}

\begin{karte}{Was ist der Unterschied zwischen einer herkömmlichen Güte (einer Reaktanz) und einer Kreisgüte?}
	\textbf{Definitionen:}\\
	\begin{minipage}[t]{0.48\textwidth}
		Für eine Reaktanz:
		\begin{equation*}
			Q = \frac{|X|}{R}
		\end{equation*}
		Beispiel Spule:
		\begin{equation*}
			Q_L = \color{red}\frac{\omega L}{R}
		\end{equation*}
		\scalebox{.7}{\input{tikz/LR-Serie.tex}}
	\end{minipage}
	\begin{minipage}[t]{0.48\textwidth}
		Kreisgüte:
		\begin{equation*}
			Q = \frac{1}{B_{rel}}
		\end{equation*}
		Beispiel Serienschwingkreis:
		\begin{equation*}
			Q_s = \frac{1}{R} \sqrt{\frac{L}{C}} = \color{red}\frac{\omega_m L}{R}
		\end{equation*}
		\scalebox{.7}{\input{tikz/LRC-Serie.tex}}
	\end{minipage}

	Wenn die Güte für den Kondensator $Q_C \rightarrow \infty$ dann ist die Definition identisch!
\end{karte}

\begin{karte}{Wie ist die Güte über die Einergie  und Leistung definiert?}
	Die Güte ist folgendermassen definiert:
	\begin{equation*}
		Q=\frac{\omega W}{P} = \frac{\text{gespeicherte Energie pro Periode}}{\text{Verlustleistung}}
	\end{equation*}
	Diese Definition gilt für die herkömmliche Güte so wie auch für die Kreisgüte.
\end{karte}

\begin{karte}{Wie verhält sich ein echter Kondensator?}
	Ein Kondensator ist laut dem Ersatzschaltbild ein Serienschwingkreis und hat dementsprechend folgenden Verlauf:
	
	\begin{minipage}{0.48\textwidth}
		\scalebox{.7}{%Autor: Simon Walker
%Version: 1.0
%Datum: 19.04.2020
%Lizenz: CC BY-NC-SA

\begin{circuitikz}
	%	\HelpCords{-1}{-1}{7}{1}
	
	\draw
	%Knoten
	(0, 0) coordinate (n1)
	(2, 0) coordinate (n2)
	(4, 0) coordinate (n3)
	(6, 0) coordinate (n4)
	
	%Nur zu Hilfszwecken
	%	(n1) node[above] {$n1$}
	%	(n2) node[above] {$n2$}
	%	(n3) node[above] {$n3$}
	%	(n4) node[above] {$n4$}
	
	(n1) to[C=$C$, o-] (n2) to [R=$R(\omega)$] (n3) 
	to [american inductor=$L$, -o] (n4)
	;
\end{circuitikz}
}
	\end{minipage}
	\begin{minipage}{0.48\textwidth}
		%Autor: Simon Walker
%Version: 1.0
%Datum: 19.04.2020
%Lizenz: CC BY-NC-SA

\begin{tikzpicture}[smooth, xscale=0.55, yscale=0.55]
	% Achsen
	\draw[->, thick] (-0.2,0) -- (5.2,0) node[right] {$\scriptstyle log \, \omega$}; % Horizontal
	\draw[->, thick] (0,-0.5) -- (0,2.7) node[above] {$ \scriptstyle \color{red} \left|Z\right|$}; % Vertikal
	
	%Impedanzfunktion
	\draw [red, thick] plot [smooth] coordinates{
		(0.2, 2.5) (2.5, 0.3) (4.8, 2.5)
	};

	%Beschriftungen
	\draw [blue, dashed] (2.8, 0.3) -- (-0.2, 0.3)
	node[left] {$\scriptstyle ESR$};
	
	
\end{tikzpicture}

	\end{minipage}\\[5pt]
	Das Minimum der Impedanzfunktion ist rein reell und wird ESR (Equivalent Series Resistance) genant. Es enspricht dem Wert des Seriewiderstands bei der entsprechenden Frequenz.
\end{karte}

\begin{karte}{Wie verhält sich ein echten Widerstand?}
	Das Ersatzschaltbild eines Widerstands hat entweder eine Spule in Serie (für  kleine $R$) oder einen Kondensator Parallel (für grosse $R$).
	
	\begin{minipage}{0.48\textwidth}
		\scalebox{.85}{%Autor: Simon Walker
%Version: 1.0
%Datum: 19.04.2020
%Lizenz: CC BY-NC-SA

\begin{circuitikz}
%	\HelpCords{-1}{-3}{5}{1}
	
	\draw
	%Knoten
	(0, -2) coordinate (n11)
	(1, -2) coordinate (n12)
	(1, -2.75) coordinate (n13)
	(1, -1.25) coordinate (n14)
	(3, -2.75) coordinate (n15)
	(3, -1.25) coordinate (n16)
	(3, -2) coordinate (n17)
	(4, -2) coordinate (n18)
	
	(0, 0) coordinate (n21)
	(2, 0) coordinate (n22)
	(4, 0) coordinate (n23)
	
	
	%Nur zu Hilfszwecken
%	(n11) node[above] {$n11$}
%	(n12) node[above] {$n12$}
%	(n13) node[above] {$n13$}
%	(n14) node[above] {$n14$}
%	(n15) node[above] {$n15$}
%	(n16) node[above] {$n16$}
%	(n17) node[above] {$n17$}
%	(n18) node[above] {$n18$}
	
%	(n21) node[above] {$n21$}
%	(n22) node[above] {$n22$}
%	(n23) node[above] {$n23$}
	;
	%Serie L
	\draw[red]
	(n21) to[R=$R$, o-, color=red] (n22) 
	to [american inductor=$L$, -o, color=red] (n23)
	;
	
	
	%Parallel C
	\draw[blue]
	(n11) to[short, o-*, color=blue] (n12)
	(n13) to[short, color=blue] (n14)
	(n14) to[R=$R$, color=blue] (n16)
	(n13) to[C=$C$, color=blue] (n15)
	(n16) to[short, color=blue] (n15)
	(n17) to[short, *-o, color=blue] (n18)
	;
\end{circuitikz}
}
	\end{minipage}
	\begin{minipage}{0.48\textwidth}
		%Autor: Simon Walker
%Version: 1.0
%Datum: 19.04.2020
%Lizenz: CC BY-NC-SA

\begin{tikzpicture}[smooth, xscale=0.7, yscale=0.7]
	% Achsen
	\draw[->, thick] (-0.2,0) -- (5.2,0) node[right] {$\scriptstyle log \, \omega$}; % Horizontal
	\draw[->, thick] (0,-0.5) -- (0,2.7) node[above] {$ \scriptstyle \left|Z\right|$}; % Vertikal
	
	%Impedanzfunktion Serie (kleines R)
	\draw [red, thick, rounded corners=8mm] 
		(0, 0.5) -- (3.5, 0.5) -- (4.8, 2.5);
		
	%Impedanzfunktion Parallel (grosses R)
	\draw [blue, thick, rounded corners=8mm] 
	(0, 2.3) -- (3.5, 2.3) -- (4.8, 0.5);
	
\end{tikzpicture}

	\end{minipage}\\[5pt]
\end{karte}

\begin{karte}{Wie verhält sich eine echte Spule?}
	Das Ersatzschaltbild einer Spule ist ein Parallelschwingkreis. Deshalb hat der Impedanzverlauf folgende Eigenschaften.
	
	\begin{minipage}{0.48\textwidth}
		%Autor: Simon Walker
%Version: 1.0
%Datum: 19.04.2020
%Lizenz: CC BY-NC-SA

\begin{circuitikz}
%	\HelpCords{-1}{-1}{5}{1}
	
	\draw
	%Knoten
	(0, 0) coordinate (n1)
	(1, 0) coordinate (n2)
	(1, -0.75) coordinate (n3)
	(1, 0.75) coordinate (n4)
	(3, -0.75) coordinate (n5)
	(3, 0.75) coordinate (n6)
	(3, 0) coordinate (n7)
	(4, 0) coordinate (n8)
	
	
	%Nur zu Hilfszwecken
%	(n1) node[above] {$n1$}
%	(n2) node[above] {$n2$}
%	(n3) node[above] {$n3$}
%	(n4) node[above] {$n4$}
%	(n5) node[above] {$n5$}
%	(n6) node[above] {$n6$}
%	(n7) node[above] {$n7$}
%	(n8) node[above] {$n8$}
	
	;	


	\draw
	(n1) to[short, o-*] (n2)
	(n3) to[short] (n4)
	(n4) to[american inductor=$L$] (n6)
	(n3) to[C=$C$] (n5)
	(n6) to[short] (n5)
	(n7) to[short, *-o] (n8)
	;
\end{circuitikz}

	\end{minipage}
	\begin{minipage}{0.48\textwidth}
		%Autor: Simon Walker
%Version: 1.0
%Datum: 19.04.2020
%Lizenz: CC BY-NC-SA

\begin{tikzpicture}[smooth, xscale=0.7, yscale=0.7]
	% Achsen
	\draw[->, thick] (-0.2,0) -- (5.2,0) node[right] {$\scriptstyle log \, \omega$}; % Horizontal
	\draw[->, thick] (0,-0.5) -- (0,2.7) node[above] {$ \scriptstyle \left|Z\right|$}; % Vertikal
	
		%Impedanzfunktion
	\draw [red, thick, rounded corners=2mm] 
		(0.2, 0.5) -- (2.5, 2.5) -- (4.8, 0.5);
	
\end{tikzpicture}

	\end{minipage}\\[5pt]
\end{karte}

	\kommentar{Reaktanzeintore (RET)}

\begin{karte}{Was ist ein Reaktanzeintor?}
	Ein Reaktanzeintor hat zwei Anschlüsse, besteht nur aus Reaktanzen und hat somit keine Wirkwiderstände.
	\\[10pt]
	\begin{minipage}{0.32\textwidth}
		Beispiel:
		\scalebox{.8}{%Autor: Simon Walker
%Version: 1.0
%Datum: 15.04.2020
%Lizenz: CC BY-NC-SA

\begin{circuitikz}
	%\HelpCords{-2}{2}{5}{-5}
	
	\draw
	%Knoten
	(2.3, 0) coordinate (n1)
	(2.3, -2) coordinate (n2)
	($(n1)+(0.7,0)$) coordinate (n11)
	($(n2)+(0.7,0)$) coordinate (n22)
	
	%Nur zu Hilfszwecken
%	(n1) node[above] {$n1$}
%	(n2) node[above] {$n2$}
%	(n11) node[above] {$n11$}
%	(n22) node[above] {$n22$}
	
	
	(0, 0) to [american inductor, o-] (n1)
	(n1) to[C, *-*] (n2)
	(n1) -- (n11) to[american inductor] (n22) -- (n2)
	(n2) to[short, -o] (0, -2)
	;
\end{circuitikz}}
	\end{minipage}
	\begin{minipage}{0.65\textwidth}
		Es wird zwischen minimalen und nicht minimalen Eintoren unterschieden. Nicht minimale Eintore können auf ein minimales Eintor reduziert werden ohne das die Änderung von aussen bemerkt wird.
	\end{minipage}\\[10pt]
	Da die RET nur aus $L$ und $C$ bestehen hat das RET bei $\omega = 0$ und bei $\omega \rightarrow \infty$ jeweils eine Pol oder eine Nullstelle.
\end{karte}

\begin{karte}{Wie sieht der Reaktanzverlauf des folgenden RET aus und was für ein Typ hat das RET?\\[5pt]
	%Autor: Simon Walker
%Version: 1.0
%Datum: 15.04.2020
%Lizenz: CC BY-NC-SA

\begin{circuitikz}
	%\HelpCords{-2}{2}{5}{-5}
	
	\draw
	%Knoten
	(2.3, 0) coordinate (n1)
	(2.3, -2) coordinate (n2)
	($(n1)+(0.7,0)$) coordinate (n11)
	($(n2)+(0.7,0)$) coordinate (n22)
	
	%Nur zu Hilfszwecken
%	(n1) node[above] {$n1$}
%	(n2) node[above] {$n2$}
%	(n11) node[above] {$n11$}
%	(n22) node[above] {$n22$}
	
	
	(0, 0) to [american inductor, o-] (n1)
	(n1) to[C, *-*] (n2)
	(n1) -- (n11) to[american inductor] (n22) -- (n2)
	(n2) to[short, -o] (0, -2)
	;
\end{circuitikz}}
	\begin{minipage}{0.55\textwidth}
		%Autor: Jürg Rast
%Datum: 04.04.2012
%Lizenz: CC BY-SA
%Grundversion von https://github.com/jrast/ELT4-Notizen/blob/master/tikzPictures/RET/ReaktanzLTyp.tex

%Geändert von: Simon Walker
%Datum: 15.04.2020
%Lizenz: CC BY-NC-SA


\usepgflibrary{shapes.misc}
\begin{tikzpicture}[smooth, xscale=0.5, yscale=0.5]
% Achsen
\draw[->, thick] (-0.2,0) -- +(8.4,0) node[right] {$\omega$}; % Horizontal
\draw[->, thick] (0,-2) -- +(0,4) node[above] {$X(\omega)$}; % Vertikal

% Plots
\draw[color=green!70!black, thick] plot[domain=0:2] (\x,{\x/8+tan(\x*3/4 r)/8}); % Erster Tan
%\draw[color=green!70!black, thick] plot[domain=2.01:4.23] (\x,{tan(\x r)}); % zweiter Tan
\draw[color=green!70!black, thick] plot[domain=2.4:8] (\x,{ (\x/3.2) -3.8/(\x -1.01) }); % letzte kurve

% Poolstellen
\draw[dashed, thick, draw=red] (2.1,-2.1) -- +(0,4.2); % Poolstelle 1
\node[cross out, draw=red, thick] (wr1) at (2.1,0) {};


\draw[dashed] plot[domain=0:8] (\x, {\x/4});
\node (wL) at (6.1,2.4) {$\omega \rightarrow \infty$};

% Nullstellen
\node[rounded rectangle, draw=blue, thick] at(0,0) {};
\node[rounded rectangle, draw=blue, thick] at(4,0) {};
\end{tikzpicture}
	\end{minipage}
	\begin{minipage}{0.4\textwidth}
		Bei diesem Beispiel handelt sich um ein L-Typ
	\end{minipage}

	Als erstes wird das Verhalten des RET bei DC und bei sehr grossen Frequenzen analysiert. Für DC leiten beide Induktivitäten also ist dort eine NS. Für hohe Frequenzen sperrt die Erste Spule also ist dort eine PS.\\
	Da es bei dieser Schaltung um ein minimales RET handelt und aus drei Bauteilen besteht, hat der Reaktanzverlauf drei Pol/Null stellen. Diese treten abwechselnd auf.
\end{karte}

\begin{karte}{Wie sieht der Reaktanzverlauf des folgenden RET aus?\\[5pt]
		%Autor: Simon Walker
%Version: 1.0
%Datum: 15.04.2020
%Lizenz: CC BY-NC-SA

\begin{circuitikz}
	%\HelpCords{-2}{2}{5}{-5}
	
	\draw
	%Knoten
	(3, 0) coordinate (n1)
	(3, -2) coordinate (n2)
	($(n1)+(0.7,0)$) coordinate (n11)
	($(n2)+(0.7,0)$) coordinate (n22)
	(0.7, 0) coordinate (n3)
	(0.7, -2) coordinate (n4)
	(0, 0) coordinate (n5)
	(0, -2) coordinate (n6)
	
	
	%Nur zu Hilfszwecken
%	(n1) node[above] {$n1$}
%	(n2) node[above] {$n2$}
%	(n11) node[above] {$n11$}
%	(n22) node[above] {$n22$}
%	(n3) node[above] {$n3$}
%	(n4) node[above] {$n4$}
%	(n5) node[above] {$n5$}
%	(n6) node[above] {$n6$}
	
	(n5) to[short,o-] (n3)
	(n3) to[C, *-*] (n4)
	(n3) to [american inductor] (n1)
	(n1) to[C, *-*] (n2)
	(n1) -- (n11) to[american inductor] (n22) -- (n2)
	(n2) to[short, -o] (n6)
	;
\end{circuitikz}}
	\centering
	%Autor: Jürg Rast
%Datum: 04.04.2012
%Lizenz: CC BY-SA
%Grundversion von https://github.com/jrast/ELT4-Notizen/blob/master/tikzPictures/RET/ReaktanzPTyp.tex

%Geändert von: Simon Walker
%Datum: 15.04.2020
%Lizenz: CC BY-NC-SA

\usepgflibrary{shapes.misc}
\begin{tikzpicture}[smooth, xscale=0.7, yscale=0.5]
% Achsen
\draw[->, thick] (-0.2,0) -- +(6.2,0) node[right] {$\omega$}; % Horizontal
\draw[->, thick] (0,-2) -- +(0,4) node[above] {$X(\omega)$}; % Vertikal

% Plots
\draw[color=green!70!black, thick] plot[domain=0.01:0.9] (\x, {tan( (\x *1.2) r )        }); % Erster Tan
\draw[color=green!70!black, thick] plot[domain=1.3:3.15] (\x,    {tan( (\x*1.2 -2.7) r)    }); % Erster Tan
\draw[color=green!70!black, thick] plot[domain=6.5:8.6] (\x-2.8,  { ((0.25 * \x) -1/(\x -6)) -2 }); % letzte kurve

% Poolstellen
\draw[dashed, thick, draw=red] (1,-2.1) -- +(0,4.2); % Poolstelle 1
\node[cross out, draw=red, thick] at (1,0) {};

\draw[dashed, thick, draw=red] (3.4,-2.1) -- +(0,4.2); % Poolstelle 2
\node[cross out, draw=red, thick] at (3.4,0) {};



% Nullstellen
\node[rounded rectangle, draw=blue, thick] at(0,0) {};
\node[rounded rectangle, draw=blue, thick] at(2.25,0) {};

\end{tikzpicture}\\
	\vspace{-5pt}
	\flushleft
	Als erstes wird das Verhalten des RET bei DC und bei sehr grossen Frequenzen analysiert. Für DC leiten beide Induktivitäten also ist dort eine NS. Für hohe Frequenzen leiten Beide Kapazitäten also ist dort ebenfalls NS.\\
	Da es bei dieser Schaltung um ein minimales RET handelt und aus vier Bauteilen besteht, hat der Reaktanzverlauf vier Pol/Null stellen. Diese treten abwechselnd auf.
\end{karte}

\begin{karte}{Ist folgendes RET ein minimales Eintor?\\[5pt]
		\input{tikz/RET3.tex}}
	Nein! Denn es hat ein Ring aus Induktivitäten. In diesem Ring kann ein Strom fliesen. Dieser Strom ist von aussen nicht sichtbar. Eine Induktivität kann ersatzlos gestrichen werden, wenn dementsprechend die anderen Induktivitäten angepasst werden. Danach ist von aussen kein Unterschied feststellbar.\\
	\begin{center}
		\input{tikz/RET3Ring.tex}
	\end{center}
\end{karte}

\begin{karte}{Welche Eigenschaften eines RET sind in der Praxis am unwahrscheinlichsten?}
	Es gibt zwei Probleme:
	\begin{compactitem}
		\item Impedanz wird nie 0 erreichen
		\item Die Impedanz kann nicht undendlich gross werden.
	\end{compactitem}
	Allerdings wird die Impedanz sich nicht komplett anders verhalten als die Reaktanz.
	%Autor: Simon Walker
%Version: 1.0
%Datum: 18.04.2020
%Lizenz: CC BY-NC-SA

\begin{tikzpicture}[smooth, xscale=0.7, yscale=0.7]
% Achsen
\draw[->, thick] (-0.2,0) -- +(5.2,0) node[right] {$\omega$}; % Horizontal
\draw[->, thick] (0,-2) -- +(0,4) node[above] {$\color{red}X \color{black}, \color{green} \left|Z\right|$}; % Vertikal

% Plots
\draw[color=red, thick] plot[domain=0.01:0.9] (\x*1.5,{tan((\x *1.2) r )}); % Erster Tan
\draw[color=red, thick] plot[domain=1.3:3.15] (\x*1.5,{tan((\x*1.2 -2.7) r)}); % Zweiter Tan
%Impedanz
%\draw[color=green, thick] plot[domain=0.01:0.9] (\x*1.5,{0.3+0.6*tan((\x *1.2) r )}); % Erster Tan

\draw [green, thick] plot [smooth] coordinates{
(0, 0.3) (0.5, 0.5) (1, 1) 
(1.5, 1.8)  (2, 1) (2.5, 0.5)
(3, 0.3) (3.5, 0.3) (4, 0.5) 
(4.5, 1) (5, 1.6) (5.5, 1.8)};


\draw [blue, thick] (3.34, 0) circle[radius=0.25];
\draw [blue, thick] (0, 0) circle[radius=0.25];
\draw [brown, thick] (1.35, 1.85) circle[radius=0.25];
\draw [brown, thick] (1.95, -2.15) circle[radius=0.25];
\draw [brown, thick] (4.73, 1.9) circle[radius=0.25];

\node [below] at (3.8, -0.2) {$\scriptstyle \color{blue} \left|Z\right| \ne 0 \Omega$};
\node at (3, 2) {$\scriptstyle \color{brown} \left|Z\right| \not\rightarrow \infty$};

\end{tikzpicture}

\end{karte}



	\kommentar{Leitungen}

\begin{karte}{Komplettiere folgendes Schema wenn gilt $\Gamma = 0.6$\\($U_{1-}$, $I_{1-}$, $P_{1-}$, $U_{2+}$, $I_{2+}$, $P_{2+}$):\\[2pt]
	\scalebox{0.62}{%Autor: Simon Walker
%Version: 1.0
%Datum: 16.07.2020
%Lizenz: CC BY-NC-SA

\begin{tikzpicture}[xscale = 0.8, yscale = 0.8]
	
	\coordinate (a1) at (0, 5);
	\coordinate (a2) at (1, 5);
	\coordinate (a3) at (4, 5);
	\coordinate (a4) at (6, 5);
	\coordinate (a5) at (8, 5);
	\coordinate (a6) at (11, 5);
	\coordinate (a7) at (12, 5);
	
	\coordinate (b1) at (0, 1);
	\coordinate (b2) at (1, 1);
	\coordinate (b3) at (4, 1);
	\coordinate (b4) at (6, 1);
	\coordinate (b5) at (8, 1);
	\coordinate (b6) at (11, 1);
	\coordinate (b7) at (12, 1);
	
	%Linien
	\draw[thick] (a1) -- (a2) (a3) -- (a5) (a6) -- (a7);
	\draw[thick] (b1) -- (b2) (b3) -- (b5) (b6) -- (b7);
	
	%Anschlüsse
	\draw[fill=white] (a1) circle (0.07);
	\draw[fill=white] (a4) circle (0.07);
	\draw[fill=white] (a7) circle (0.07);
	\draw[fill=white] (b1) circle (0.07);
	\draw[fill=white] (b4) circle (0.07);
	\draw[fill=white] (b7) circle (0.07);
	
	%Leitungen
	\draw[thick] ($(a2)-(0, 0.25)$) rectangle ($(a3)+(0, 0.25)$);
	\draw[thick] ($(a5)-(0, 0.25)$) rectangle ($(a6)+(0, 0.25)$);
	\draw[thick] ($(b2)-(0, 0.25)$) rectangle ($(b3)+(0, 0.25)$);
	\draw[thick] ($(b5)-(0, 0.25)$) rectangle ($(b6)+(0, 0.25)$);
	
	%gestrichelte Linie
	\draw[dashed, thick] ($(a4)+(0, 1)$) -- ($(b4)-(0, 1)$);
	
	%Wellen Pfeile (Spannung)
	\draw [blue] ($(a2)!0.5!(a3)$)++ (-0.8, 0.5) -- ++(0.3,0)
	coordinate (p);
	\begin{scope}[shift={(p)}]
	\draw[blue, ->] plot[domain=0:540] ({1/540*\x}, {sin(\x)/8}) -- ++(0.3, 0);
	\end{scope}
	\node[blue] at ($(a2)!0.5!(a3)+(0,1)$) {$U_{1+} = 1V$};
	 
	\draw [blue] ($(a5)!0.5!(a6)$)++ (-0.8, 0.5) -- ++(0.3,0) coordinate (p);
	\begin{scope}[shift={(p)}]
	\draw[blue, ->] plot[domain=0:540] ({1/540*\x}, {sin(\x)/8}) -- ++(0.3, 0);
	\end{scope}
	\node[blue] at ($(a5)!0.5!(a6)+(0,1)$) {$U_{2+}= ?$};
	 
	\draw [blue, <-] ($(b2)!0.5!(b3)$)++ (-0.8, -0.5) -- ++(0.3,0) coordinate (p);
	\begin{scope}[shift={(p)}]
	\draw[blue] plot[domain=0:540] ({1/540*\x}, {sin(\x)/8}) -- ++(0.3, 0);
	\end{scope}
	\node[blue] at ($(b2)!0.5!(b3)-(0,1)$) {$U_{1-}= ?$};
	 
	 %Wellen Pfeile (Strom)
	 \draw [red] ($(a2)!0.5!(a3)$)++ (-0.8, -0.5) -- ++(0.3,0) coordinate (p);
	\begin{scope}[shift={(p)}]
	\draw[red, ->] plot[domain=0:540] ({1/540*\x}, {sin(\x)/8}) -- ++(0.3, 0);
	\end{scope}
	\node[red] at ($(a2)!0.5!(a3)-(0,1)$) {$I_{1+} = 1A$};
	 
	\draw [red] ($(a5)!0.5!(a6)$)++ (-0.8, -0.5) -- ++(0.3,0) coordinate (p);
	\begin{scope}[shift={(p)}]
	\draw[red, ->] plot[domain=0:540] ({1/540*\x}, {sin(\x)/8}) -- ++(0.3, 0);
	\end{scope}
	\node[red] at ($(a5)!0.5!(a6)-(0,1)$) {$I_{2+}= ?$};
	 
	\draw [red] ($(b2)!0.5!(b3)$)++ (-0.8, +0.5) -- ++(0.3,0) coordinate (p);
	\begin{scope}[shift={(p)}]
	\draw[red, ->] plot[domain=0:540] ({1/540*\x}, {sin(\x)/8})-- ++(0.3, 0);
	\end{scope}
	\node[red] at ($(b2)!0.5!(b3)+(0,1)$) {$I_{1-}= ?$};
	 
	%Leistungspfeile
	\draw[very thick, ->, brown] ($(a1)!0.5!(b1)+(0,-0.5)$) -- node[above] {$P_{1+} = 1W$} ++(1.5,0);
	
	\draw[very thick, ->, brown] ($(a4)!0.5!(b4)+(-0.4,-0.5)$) -- node[above] {$P_{1-} = ?$} ++(-1.5,0);
	\draw[very thick, ->, brown] ($(a4)!0.5!(b4)+(0.4,-0.5)$) -- node[above] {$P_{2+}= ?$} ++(1.5,0);
	
\end{tikzpicture}}}
	\begin{align*}
		U_{1-} &= \Gamma \cdot U_{1+} = \mathbf{0.6V}\\
		I_{1-} &= \Gamma \cdot I_{1+} = \mathbf{0.6A}\\
		P_{1-} &= U_{1-} \cdot I_{1-} = \Gamma^2 \cdot P_{1+} = \mathbf{0.36W}\\
		&\\
		U_{2+} &= (1+\Gamma) \cdot U_{1+} = \mathbf{1.6V}\\
		I_{2+} &= (1-\Gamma) \cdot I_{1+} = \mathbf{0.4A}\\
		P_{2+} &= (1+\Gamma) \cdot (1-\Gamma) = (1 - \Gamma^2) \cdot P_{1+} = \mathbf{0.64W}
	\end{align*}
\end{karte}

	\kommentar{Zweitore}

\begin{karte}{Beschreibe die Impedanzmatrix $\mathbf{Z}$. Wie werden die einzelne Elemente gemessen/berechnet?\\
	\begin{equation*}
		\left[\begin{array}{c}
		U_1 \\ U_2
		\end{array}\right]
		= 
		\underbrace{\left[\begin{array}{cc}
			Z_{11} & Z_{12}\\
			Z_{21} & Z_{22}
		\end{array}\right]}_{\mathbf{Z}}
		\left[\begin{array}{c}
		I_1 \\ I_2
		\end{array}\right]
	\end{equation*}}
	\renewcommand*{\arraystretch}{2.2}
	\begin{tabular}{lcc}	
		$\bullet$ Leerlauf-Eingangsimpedanz & \qquad & $Z_{11} = \left.\dfrac{U_1}{I_1}\right|_{I_2=0}$\\
		$\bullet$ Eingangs-Koppel-/Kernimpedanz & \qquad & $Z_{12} = \left.\dfrac{U_1}{I_2}\right|_{I_1=0}$\\
		$\bullet$ Ausgangs-Koppel-/Kernimpedanz & \qquad & $Z_{21} = \left.\dfrac{U_2}{I_1}\right|_{I_2=0}$\\
		$\bullet$ Leerlauf-Ausgangsimpedanz & \qquad & $Z_{11} = \left.\dfrac{U_2}{I_2}\right|_{I_1=0}$\\
	\end{tabular}
\end{karte}

\begin{karte}{Beschreibe die Admittanzmatrix $\mathbf{Y}$. Wie werden die einzelne Elemente gemessen/berechnet?\\
		\begin{equation*}
		\left[\begin{array}{c}
		I_1 \\ I_2
		\end{array}\right]
		=
		\underbrace{\left[\begin{array}{cc}
		Y_{11} & Y_{12}\\
		Y_{21} & Y_{22}
		\end{array}\right]}_{\mathbf{Y}}
		\left[\begin{array}{c}
		U_1 \\ U_2
		\end{array}\right]
		\end{equation*}}
	\renewcommand*{\arraystretch}{2.2}
	\begin{tabular}{lcc}	
		$\bullet$ Kurzschluss-Eingangsadmittanz & \qquad & $Y_{11} = \left.\dfrac{I_1}{U_1}\right|_{U_2=0}$\\
		$\bullet$ Rückwirkungsleiterwert & \qquad & $Y_{12} = \left.\dfrac{I_1}{U_2}\right|_{U_1=0}$\\
		$\bullet$ Steilheit & \qquad & $Y_{21} = \left.\dfrac{I_2}{U_1}\right|_{U_2=0}$\\
		$\bullet$ Kurzschluss-Ausgangsadmittanz & \qquad & $Y_{11} = \left.\dfrac{I_2}{U_2}\right|_{U_1=0}$\\
	\end{tabular}\\[3pt]
	Die Admittanz Matrix lässt sich auch über die Impedanzmatrix berechnen:
	\begin{tightcenter}
		$\mathbf{Y} = \mathbf{Z}^{-1}$
	\end{tightcenter}
\end{karte}

\begin{karte}{Beschreibe die Kettenmatrix $\mathbf{A}$. Wie werden die einzelne Elemente gemessen/berechnet?\\
		\begin{equation*}
		\left[\begin{array}{c}
		U_1 \\ I_1
		\end{array}\right]
		= 
		\underbrace{\left[\begin{array}{cc}
			A_{11} & A_{12}\\
			A_{21} & A_{22}
			\end{array}\right]}_{\mathbf{A}}
		\left[\begin{array}{c}
		U_2 \\ -I_2
		\end{array}\right]
		\end{equation*}}
	\renewcommand*{\arraystretch}{2.2}
	{\small
	\begin{tabular}{ll}	
		$\bullet$ Reziproke Leerlauf-Spannungsübersetzung & $A_{11} = \left.\dfrac{U_1}{U_2}\right|_{I_2=0}$\\
		$\bullet$ Kurzschluss-Kernimpedanz vorwärts & $A_{12} = \left.\dfrac{U_1}{-I_2}\right|_{U_2=0}$\\
		$\bullet$ Leerlauf-Kernadmittanz vorwärts & $A_{21} = \left.\dfrac{I_1}{U_2}\right|_{I_2=0}$\\
		$\bullet$ Reziproke Kurzschluss-Stromübersetzung & $A_{11} = \left.\dfrac{I_1}{-I_2}\right|_{U_2=0}$\\
	\end{tabular}
	}
\end{karte}

\begin{karte}{Beschreibe die Hybridmatrix $\mathbf{H}$. Wie werden die einzelne Elemente gemessen/berechnet?\\
		\begin{equation*}
		\left[\begin{array}{c}
		U_1 \\ I_2
		\end{array}\right]
		= 
		\underbrace{\left[\begin{array}{cc}
			H_{11} & H_{12}\\
			H_{21} & H_{22}
			\end{array}\right]}_{\mathbf{H}}
		\left[\begin{array}{c}
		I_1 \\ U_2
		\end{array}\right]
		\end{equation*}}
	\renewcommand*{\arraystretch}{2.2}
	{\small 
	\begin{tabular}{ll}	
		$\bullet$ Kurzschluss-Eingangsimpedanz & $H_{11} = \left.\dfrac{U_1}{I_1}\right|_{U_2=0}$\\
		$\bullet$ Leerlauf-Spannungsrückwirkung & $H_{12} = \left.\dfrac{U_1}{U_2}\right|_{I_1=0}$\\
		$\bullet$ (Negative) Kurzschluss-Stromverstärkung & $H_{21} = \left.\dfrac{I_2}{I_1}\right|_{U_2=0}$\\
		$\bullet$ Leerlauf-Ausgangsadmittanz & $H_{11} = \left.\dfrac{I_2}{U_2}\right|_{I_1=0}$\\
	\end{tabular}
	}
\end{karte}

	\kommentar{EMV}

\begin{karte}{Wie lassen sich kapazitiv eingekoppelte Störungen reduzieren?}
	Es gibt fünf Ansätze:\\
	$\bullet$ \textbf{Koppelkapazitäten minimieren:}
	\begin{compactitem}
		\item kurze Verbindungsleitungen
		\item grosser Abstand zwischen den betreffenden Leitungen
		\item Vermeidung von parallel geführten Leitungen
	\end{compactitem}
	$\bullet$ \textbf{Niederohmige Speisung:}
	\begin{compactitem}
		\item Signalspannungsquelle mit möglichst geringem $R_i$
		\item Verringern von $du/dt$ der Störspannung
	\end{compactitem}
	$\bullet$ \textbf{Abschirmung}\\
	$\bullet$ \textbf{Differentielle Signalübertragung}\\
	$\bullet$ \textbf{Verringerung der Flankensteilheiten}
\end{karte}

\begin{karte}{Wie lassen sich induktiv eingekoppelte Störungen reduzieren?}
	\begin{compactitem}
		\item Verringerung der Gegeninduktivität $M$ (maximieren des Abstands, minimieren der Schleifengrössen)
		\item Verdrillen von Hin- und Rückleitern
		\item Symmetrische (differentielle) Signalübertragung
		\item Reduzierung der Änderungsgeschwindigkeit des Störstromes
		\item Herabsetzen der Flussänderung $d \Phi / d t$ durch eine Kurzschlussschleife in unmittelbarer Nähe des gefährdeten Nutzkreises
		\item Abschirmen der Leitungen, Stromkreise und Baugruppen durch ferro-/ferrimagnetische Schirme
	\end{compactitem}
\end{karte}

\begin{karte}{Wie lassen sich galvanisch eingekoppelte und strahlungseingekoppelte Störungen reduzieren?}
	\textbf{Galvanische Kopplung:}
	\begin{itemize}
		\item Entkopplungskapazitäten
		\item getrennte Zuleitungen
	\end{itemize}
	\vspace{5pt}
	\textbf{Strahlungseinkopplung:}
	\begin{itemize}
		\item Schirmung (nicht beliebig dünn; Skin-Tiefe muss berücksichtigt werden)
	\end{itemize}
\end{karte}

	
\end{document}
