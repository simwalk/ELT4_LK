%
%    Autor: Simon Walker
%			simon.walker@hsr.ch
%	Lizenz: CC BY-NC-SA
%			Für bilder gelten eventuel andere Lizenzen dise sind jeweils dann als Kommentar hinter dem Include angegeben


%	 Dieses LaTeX Dokument benutzt eine Karteikartenklasse welche von Ronny Bergmann <mail@rbergmann.info> entwickelt wurde version 1.8b
%	 Die Dokumentation und die Karteikartenklasse sind unter folgendem Link erhältlich
%    https://github.com/kellertuer/Kartei

%------------------------------------------------------------
% Lernkarten zum Fach ELT4,
%------------------------------------------------------------

\documentclass[a7paper,10pt,grid=none% %Grösse der Karteikarten auswählen (A5-A9)
%,toc			%Kommentarlöschen falls eine Zusammenstellung gewünscht wird
%,print			%Kommentar löschen falls eine Ausdruckversion erstellt werden soll
]{kartei}

\usepackage[utf8]{inputenc} %UTF8
\usepackage{hyperref}

\usepackage{amsmath}
\usepackage{amssymb} %Für Real und Complex Symbol
\usepackage{bm}
\usepackage{paralist} % für compactenum and compactitem
\usepackage{xcolor} %Für Farbige Mathesymbole

\usepackage[T1]{fontenc}% wichtig für Trennung von Wörtern mit Umlauten

\usepackage[ngerman]{babel}% deutsche Trennregeln
\usepackage{microtype}% verbesserter Randausgleich

\usepackage{hyphsubst}
\usepackage{lmodern} %Schriftart

\setlength{\parindent}{0em} %Einrücken verhindern


\usepackage{tikz}
\usetikzlibrary{math}
\usepackage{xstring} %If statements
\usepackage{scalefnt}
\usepackage[european]{circuitikz}

%\usepackage{cancel} %\cancel befehl kürzen %Farbe ändern mit \renewcommand\CancelColor{\color{red}}

\newenvironment{tightcenter}{%
	\setlength\topsep{0pt}
	\setlength\parskip{0pt}
	\begin{center}
	}{%
	\end{center}
}


\usepackage{pgfplots}
\usepgfplotslibrary{fillbetween}

\usepgfplotslibrary{smithchart}
\pgfplotsset{compat=1.11}

%\input{tikz/Electrical_Components.tex}
%Autor: Simon Walker
%Version: 1.0
%Datum: 16.12.2019
%Eigene" Tikz befehle


\newcommand\markangle[6][red]{% [color] {X} {origin} {Y} {mark} {radius}
	% filled circle: red by default
	\begin{scope}
		\path[clip] (#2) -- (#3) -- (#4);
		\draw[color=#1,draw=#1,name path=circle] %opacity=0.5
		(#3) circle (#6mm);
	\end{scope}
	% middle calculation
	\path[name path=line one] (#3) -- (#2);
	\path[name path=line two] (#3) -- (#4);
	\path[%
	name intersections={of=line one and circle, by={inter one}},
	name intersections={of=line two and circle, by={inter two}}
	] (inter one) -- (inter two) coordinate[pos=.5] (middle);
	% bissectrice definition
	\path[%
	name path=bissectrice
	] (#3) -- (barycentric cs:#3=-1,middle=1.2);
	% put mark
	\path[
	name intersections={of=bissectrice and circle, by={middleArc}}
	] (#3) -- (middleArc) node[pos=0.7] {#5}; % node[pos=1.3]
}

\newcommand{\HelpCords}[4]{
	\draw [help lines] (#1,#2) grid (#3,#4);
	\foreach \i in {#1,..., #3}
	\node [below] at (\i,#2) {$\i$};
	\foreach \i in {#2,..., #4}
	\node [left] at (#1,\i) {$\i$};
}


%Eingener Befehl: Erzeugt zwei Minipage nebeneinander
\newcommand{\divTwo}[2]{
\begin{minipage}[t]{0.48\textwidth}
	#1	
\end{minipage}
\begin{minipage}[t]{0.48\textwidth}
	#2
\end{minipage}
}



\begin{document}
	\setcardpagelayout
	
	\fach{ELT4}
	%Karten für PreReleas
	\kommentar{Recap ELT3}

\begin{karte}{Was ist eine Reaktanz}
	Die Reaktanz ($X$) ist der Blindwiderstand. Somit ist er der Imaginärteil der Impedanz ($Z$). 
	Die Reaktanz lässt sich folgendermassen berechnen:\\[10pt]
	\begin{minipage}[t]{0.48\textwidth}
		Für Kapazitäten:\\
		$X = - \dfrac{1}{\omega C}$
	\end{minipage}
	\begin{minipage}[t]{0.48\textwidth}
		Für Induktivitäten:\\
		$X = \omega L$
	\end{minipage}\\[10pt]
	Da die Impedanz $Z = jX$ ist folgt:\\[10pt]
	\begin{minipage}[t]{0.48\textwidth}
		Für Kapazitäten:\\
		$Z = jX = - \dfrac{j}{\omega C} = \dfrac{1}{j \omega C}$
	\end{minipage}
	\begin{minipage}[t]{0.48\textwidth}
		Für Induktivitäten:\\
		$Z = jX = j \omega L$
	\end{minipage}
\end{karte}

	\kommentar{Frequenzabhängigkeit}


	\kommentar{Schwingkreise}

\begin{karte}{Was ist ein Schwingkreis und welche Betriebsmodus gibt es?}
	Ein Schwingkreis ist:\\
	\begin{itemize}
		\item ein Netzwerk aus Ohmischen und Kapazitive Bauteile
		\item ein resonanzfähiges Netzwerk
		\item Kann parallel oder seriell geschaltet werden
	\end{itemize}
	Es gibt 2 Betriebsmodus:
	\begin{itemize}
		\item Freie Schwingung 
		\item Erzwungene Schwingung (Durch Quelle angeregt)
	\end{itemize}
\end{karte}

\begin{karte}{Welche Frequenzen gibt es in einem Schwingkreis?}
	Es gibt drei verschiedene Frequenzen in einem Schwingkreis:
	\begin{itemize}
		\item Eigenfrequenz $\omega_0$\\
		(Resonanzfrequenz des verlustlosen Schwingkreis)
		\item Resonanzfrequenz $\omega_r$\\
		($Im(Z) = 0$, Reine Reelle Impedanzen und Admitanzen)
		\item Extremalfrequenz $\omega_m$\\
		(Frequenz bei welcher die Impedanz maximal oder minimal ist)
	\end{itemize}
\end{karte}

\begin{karte}{Wie ist die Bandbreite eines Schwingkreises definiert?}
	\textbf{(absolute) Bandbreite}:
	\begin{equation*}
		B = f^+ - f^- = \dfrac{\omega^+ - \omega^-}{2 \pi}
	\end{equation*}
	Die Bandbreite ist der Frequenzbereich, in welchem die Impedanz bzw. Admittanz des Resonators sich nur um den Faktor $\sqrt{2}$ vom nächsten (lokalen) Minimum $\omega_m$ unterscheidet.\\
	Oder Formal ausgedrückt: \quad $|Z (\omega^{\pm})|=\sqrt{2}\cdot|Z(\omega_{m})|$\\
	\textbf{relative Bandbreite}:
	\begin{equation*}
	B_{rel} = \frac{f^+ - f^-}{f_m} = \frac{\omega^+ - \omega^-}{\omega_m} =\frac{1}{Q_s}
	\end{equation*}
	Wobei $Q_s$ die Schwingkreisgüte ist.
\end{karte}

\begin{karte}{Was bedeutet Resonanz?}
	\begin{itemize}
		\item Aus der Akustik:\\
		Verstärktes Mitschwingen eines schwingfähigen Systems
		\item Die Resonanzfrequenz $\omega_r$ ist die Frequenz bei welcher die elektrische und magnetische Energie gleich gross sind $W_C = W_L$.
	\end{itemize}
	Bei der Resonanzfrequenz gilt folglich:
	\begin{itemize}
		\item $|X_C| = |X_L|$
		\item $Im(Z) = 0$
		\item $\tan (\varphi) = 0 \quad \rightarrow \quad \varphi = 0$
	\end{itemize}
\end{karte}

\begin{karte}{Wie verhält sich ein echter Kondensator?}
	Ein Kondensator ist laut dem Ersatzschaltbild ein Serienschwingkreis und hat dementsprechend folgenden Verlauf:
	
	\begin{minipage}{0.48\textwidth}
		\scalebox{.7}{%Autor: Simon Walker
%Version: 1.0
%Datum: 19.04.2020
%Lizenz: CC BY-NC-SA

\begin{circuitikz}
	%	\HelpCords{-1}{-1}{7}{1}
	
	\draw
	%Knoten
	(0, 0) coordinate (n1)
	(2, 0) coordinate (n2)
	(4, 0) coordinate (n3)
	(6, 0) coordinate (n4)
	
	%Nur zu Hilfszwecken
	%	(n1) node[above] {$n1$}
	%	(n2) node[above] {$n2$}
	%	(n3) node[above] {$n3$}
	%	(n4) node[above] {$n4$}
	
	(n1) to[C=$C$, o-] (n2) to [R=$R(\omega)$] (n3) 
	to [american inductor=$L$, -o] (n4)
	;
\end{circuitikz}
}
	\end{minipage}
	\begin{minipage}{0.48\textwidth}
		%Autor: Simon Walker
%Version: 1.0
%Datum: 19.04.2020
%Lizenz: CC BY-NC-SA

\begin{tikzpicture}[smooth, xscale=0.55, yscale=0.55]
	% Achsen
	\draw[->, thick] (-0.2,0) -- (5.2,0) node[right] {$\scriptstyle log \, \omega$}; % Horizontal
	\draw[->, thick] (0,-0.5) -- (0,2.7) node[above] {$ \scriptstyle \color{red} \left|Z\right|$}; % Vertikal
	
	%Impedanzfunktion
	\draw [red, thick] plot [smooth] coordinates{
		(0.2, 2.5) (2.5, 0.3) (4.8, 2.5)
	};

	%Beschriftungen
	\draw [blue, dashed] (2.8, 0.3) -- (-0.2, 0.3)
	node[left] {$\scriptstyle ESR$};
	
	
\end{tikzpicture}

	\end{minipage}\\[5pt]
	Das Minimum der Impedanzfunktion ist rein reell und wird ESR (Equivalent Series Resistance) genant. Es enspricht dem Wert des Seriewiderstands bei der entsprechenden Frequenz.
\end{karte}

\begin{karte}{Wie verhält sich ein echten Widerstand?}
	Das Ersatzschaltbild eines Widerstands hat entweder eine Spule in Serie (für  kleine $R$) oder einen Kondensator Parallel (für grosse $R$).
	
	\begin{minipage}{0.48\textwidth}
		\scalebox{.85}{%Autor: Simon Walker
%Version: 1.0
%Datum: 19.04.2020
%Lizenz: CC BY-NC-SA

\begin{circuitikz}
%	\HelpCords{-1}{-3}{5}{1}
	
	\draw
	%Knoten
	(0, -2) coordinate (n11)
	(1, -2) coordinate (n12)
	(1, -2.75) coordinate (n13)
	(1, -1.25) coordinate (n14)
	(3, -2.75) coordinate (n15)
	(3, -1.25) coordinate (n16)
	(3, -2) coordinate (n17)
	(4, -2) coordinate (n18)
	
	(0, 0) coordinate (n21)
	(2, 0) coordinate (n22)
	(4, 0) coordinate (n23)
	
	
	%Nur zu Hilfszwecken
%	(n11) node[above] {$n11$}
%	(n12) node[above] {$n12$}
%	(n13) node[above] {$n13$}
%	(n14) node[above] {$n14$}
%	(n15) node[above] {$n15$}
%	(n16) node[above] {$n16$}
%	(n17) node[above] {$n17$}
%	(n18) node[above] {$n18$}
	
%	(n21) node[above] {$n21$}
%	(n22) node[above] {$n22$}
%	(n23) node[above] {$n23$}
	;
	%Serie L
	\draw[red]
	(n21) to[R=$R$, o-, color=red] (n22) 
	to [american inductor=$L$, -o, color=red] (n23)
	;
	
	
	%Parallel C
	\draw[blue]
	(n11) to[short, o-*, color=blue] (n12)
	(n13) to[short, color=blue] (n14)
	(n14) to[R=$R$, color=blue] (n16)
	(n13) to[C=$C$, color=blue] (n15)
	(n16) to[short, color=blue] (n15)
	(n17) to[short, *-o, color=blue] (n18)
	;
\end{circuitikz}
}
	\end{minipage}
	\begin{minipage}{0.48\textwidth}
		%Autor: Simon Walker
%Version: 1.0
%Datum: 19.04.2020
%Lizenz: CC BY-NC-SA

\begin{tikzpicture}[smooth, xscale=0.7, yscale=0.7]
	% Achsen
	\draw[->, thick] (-0.2,0) -- (5.2,0) node[right] {$\scriptstyle log \, \omega$}; % Horizontal
	\draw[->, thick] (0,-0.5) -- (0,2.7) node[above] {$ \scriptstyle \left|Z\right|$}; % Vertikal
	
	%Impedanzfunktion Serie (kleines R)
	\draw [red, thick, rounded corners=8mm] 
		(0, 0.5) -- (3.5, 0.5) -- (4.8, 2.5);
		
	%Impedanzfunktion Parallel (grosses R)
	\draw [blue, thick, rounded corners=8mm] 
	(0, 2.3) -- (3.5, 2.3) -- (4.8, 0.5);
	
\end{tikzpicture}

	\end{minipage}\\[5pt]
\end{karte}

\begin{karte}{Wie verhält sich eine echte Spule?}
	Das Ersatzschaltbild einer Spule ist ein Parallelschwingkreis. Deshalb hat der Impedanzverlauf folgende Eigenschaften.
	
	\begin{minipage}{0.48\textwidth}
		%Autor: Simon Walker
%Version: 1.0
%Datum: 19.04.2020
%Lizenz: CC BY-NC-SA

\begin{circuitikz}
%	\HelpCords{-1}{-1}{5}{1}
	
	\draw
	%Knoten
	(0, 0) coordinate (n1)
	(1, 0) coordinate (n2)
	(1, -0.75) coordinate (n3)
	(1, 0.75) coordinate (n4)
	(3, -0.75) coordinate (n5)
	(3, 0.75) coordinate (n6)
	(3, 0) coordinate (n7)
	(4, 0) coordinate (n8)
	
	
	%Nur zu Hilfszwecken
%	(n1) node[above] {$n1$}
%	(n2) node[above] {$n2$}
%	(n3) node[above] {$n3$}
%	(n4) node[above] {$n4$}
%	(n5) node[above] {$n5$}
%	(n6) node[above] {$n6$}
%	(n7) node[above] {$n7$}
%	(n8) node[above] {$n8$}
	
	;	


	\draw
	(n1) to[short, o-*] (n2)
	(n3) to[short] (n4)
	(n4) to[american inductor=$L$] (n6)
	(n3) to[C=$C$] (n5)
	(n6) to[short] (n5)
	(n7) to[short, *-o] (n8)
	;
\end{circuitikz}

	\end{minipage}
	\begin{minipage}{0.48\textwidth}
		%Autor: Simon Walker
%Version: 1.0
%Datum: 19.04.2020
%Lizenz: CC BY-NC-SA

\begin{tikzpicture}[smooth, xscale=0.7, yscale=0.7]
	% Achsen
	\draw[->, thick] (-0.2,0) -- (5.2,0) node[right] {$\scriptstyle log \, \omega$}; % Horizontal
	\draw[->, thick] (0,-0.5) -- (0,2.7) node[above] {$ \scriptstyle \left|Z\right|$}; % Vertikal
	
		%Impedanzfunktion
	\draw [red, thick, rounded corners=2mm] 
		(0.2, 0.5) -- (2.5, 2.5) -- (4.8, 0.5);
	
\end{tikzpicture}

	\end{minipage}\\[5pt]
\end{karte}

	\kommentar{Reaktanzeintore (RET)}

\begin{karte}{Was ist ein Reaktanzeintor?}
	Ein Reaktanzeintor hat zwei Anschlüsse, besteht nur aus Reaktanzen und hat somit keine Wirkwiderstände.
	\flushleft
	\begin{minipage}{0.32\textwidth}
		Beispiel:
		\scalebox{.8}{%Autor: Simon Walker
%Version: 1.0
%Datum: 15.04.2020
%Lizenz: CC BY-NC-SA

\begin{circuitikz}
	%\HelpCords{-2}{2}{5}{-5}
	
	\draw
	%Knoten
	(2.3, 0) coordinate (n1)
	(2.3, -2) coordinate (n2)
	($(n1)+(0.7,0)$) coordinate (n11)
	($(n2)+(0.7,0)$) coordinate (n22)
	
	%Nur zu Hilfszwecken
%	(n1) node[above] {$n1$}
%	(n2) node[above] {$n2$}
%	(n11) node[above] {$n11$}
%	(n22) node[above] {$n22$}
	
	
	(0, 0) to [american inductor, o-] (n1)
	(n1) to[C, *-*] (n2)
	(n1) -- (n11) to[american inductor] (n22) -- (n2)
	(n2) to[short, -o] (0, -2)
	;
\end{circuitikz}}
	\end{minipage}
	\begin{minipage}{0.65\textwidth}
		Es wird zwischen minimalen und nicht minimalen Eintoren unterschieden. Nicht minimale Eintore können auf ein minimales Eintor reduziert werden ohne das die Änderung von aussen bemerkt wird.
	\end{minipage}\\[5pt]
	Da die RET nur aus $L$ und $C$ bestehen hat das RET bei $\omega = 0$ und bei $\omega \rightarrow \infty$ jeweils eine Pol oder eine Nullstelle.
\end{karte}

\begin{karte}{Wie sieht der Reaktanzverlauf des folgenden RET aus und was für ein Typ hat das RET?\\[5pt]
	%Autor: Simon Walker
%Version: 1.0
%Datum: 15.04.2020
%Lizenz: CC BY-NC-SA

\begin{circuitikz}
	%\HelpCords{-2}{2}{5}{-5}
	
	\draw
	%Knoten
	(2.3, 0) coordinate (n1)
	(2.3, -2) coordinate (n2)
	($(n1)+(0.7,0)$) coordinate (n11)
	($(n2)+(0.7,0)$) coordinate (n22)
	
	%Nur zu Hilfszwecken
%	(n1) node[above] {$n1$}
%	(n2) node[above] {$n2$}
%	(n11) node[above] {$n11$}
%	(n22) node[above] {$n22$}
	
	
	(0, 0) to [american inductor, o-] (n1)
	(n1) to[C, *-*] (n2)
	(n1) -- (n11) to[american inductor] (n22) -- (n2)
	(n2) to[short, -o] (0, -2)
	;
\end{circuitikz}}
	\flushleft
	\begin{minipage}{0.55\textwidth}
		%Autor: Jürg Rast
%Datum: 04.04.2012
%Lizenz: CC BY-SA
%Grundversion von https://github.com/jrast/ELT4-Notizen/blob/master/tikzPictures/RET/ReaktanzLTyp.tex

%Geändert von: Simon Walker
%Datum: 15.04.2020
%Lizenz: CC BY-NC-SA


\usepgflibrary{shapes.misc}
\begin{tikzpicture}[smooth, xscale=0.5, yscale=0.5]
% Achsen
\draw[->, thick] (-0.2,0) -- +(8.4,0) node[right] {$\omega$}; % Horizontal
\draw[->, thick] (0,-2) -- +(0,4) node[above] {$X(\omega)$}; % Vertikal

% Plots
\draw[color=green!70!black, thick] plot[domain=0:2] (\x,{\x/8+tan(\x*3/4 r)/8}); % Erster Tan
%\draw[color=green!70!black, thick] plot[domain=2.01:4.23] (\x,{tan(\x r)}); % zweiter Tan
\draw[color=green!70!black, thick] plot[domain=2.4:8] (\x,{ (\x/3.2) -3.8/(\x -1.01) }); % letzte kurve

% Poolstellen
\draw[dashed, thick, draw=red] (2.1,-2.1) -- +(0,4.2); % Poolstelle 1
\node[cross out, draw=red, thick] (wr1) at (2.1,0) {};


\draw[dashed] plot[domain=0:8] (\x, {\x/4});
\node (wL) at (6.1,2.4) {$\omega \rightarrow \infty$};

% Nullstellen
\node[rounded rectangle, draw=blue, thick] at(0,0) {};
\node[rounded rectangle, draw=blue, thick] at(4,0) {};
\end{tikzpicture}
	\end{minipage}
	\begin{minipage}{0.4\textwidth}
		Bei diesem Beispiel handelt sich um ein L-Typ
	\end{minipage}
	Als erstes wird das Verhalten des RET bei DC und bei sehr grossen Frequenzen analysiert. Für DC leiten beide Induktivitäten also ist dort eine NS. Für hohe Frequenzen sperrt die Erste Spule also ist dort eine PS.\\
	Da es bei dieser Schaltung um ein minimales RET handelt und aus drei Bauteilen besteht, hat der Reaktanzverlauf drei Pol/Null stellen. Diese treten abwechselnd auf.
\end{karte}

\begin{karte}{Wie sieht der Reaktanzverlauf des folgenden RET aus?\\[5pt]
		%Autor: Simon Walker
%Version: 1.0
%Datum: 15.04.2020
%Lizenz: CC BY-NC-SA

\begin{circuitikz}
	%\HelpCords{-2}{2}{5}{-5}
	
	\draw
	%Knoten
	(3, 0) coordinate (n1)
	(3, -2) coordinate (n2)
	($(n1)+(0.7,0)$) coordinate (n11)
	($(n2)+(0.7,0)$) coordinate (n22)
	(0.7, 0) coordinate (n3)
	(0.7, -2) coordinate (n4)
	(0, 0) coordinate (n5)
	(0, -2) coordinate (n6)
	
	
	%Nur zu Hilfszwecken
%	(n1) node[above] {$n1$}
%	(n2) node[above] {$n2$}
%	(n11) node[above] {$n11$}
%	(n22) node[above] {$n22$}
%	(n3) node[above] {$n3$}
%	(n4) node[above] {$n4$}
%	(n5) node[above] {$n5$}
%	(n6) node[above] {$n6$}
	
	(n5) to[short,o-] (n3)
	(n3) to[C, *-*] (n4)
	(n3) to [american inductor] (n1)
	(n1) to[C, *-*] (n2)
	(n1) -- (n11) to[american inductor] (n22) -- (n2)
	(n2) to[short, -o] (n6)
	;
\end{circuitikz}}
	\centering
	%Autor: Jürg Rast
%Datum: 04.04.2012
%Lizenz: CC BY-SA
%Grundversion von https://github.com/jrast/ELT4-Notizen/blob/master/tikzPictures/RET/ReaktanzPTyp.tex

%Geändert von: Simon Walker
%Datum: 15.04.2020
%Lizenz: CC BY-NC-SA

\usepgflibrary{shapes.misc}
\begin{tikzpicture}[smooth, xscale=0.7, yscale=0.5]
% Achsen
\draw[->, thick] (-0.2,0) -- +(6.2,0) node[right] {$\omega$}; % Horizontal
\draw[->, thick] (0,-2) -- +(0,4) node[above] {$X(\omega)$}; % Vertikal

% Plots
\draw[color=green!70!black, thick] plot[domain=0.01:0.9] (\x, {tan( (\x *1.2) r )        }); % Erster Tan
\draw[color=green!70!black, thick] plot[domain=1.3:3.15] (\x,    {tan( (\x*1.2 -2.7) r)    }); % Erster Tan
\draw[color=green!70!black, thick] plot[domain=6.5:8.6] (\x-2.8,  { ((0.25 * \x) -1/(\x -6)) -2 }); % letzte kurve

% Poolstellen
\draw[dashed, thick, draw=red] (1,-2.1) -- +(0,4.2); % Poolstelle 1
\node[cross out, draw=red, thick] at (1,0) {};

\draw[dashed, thick, draw=red] (3.4,-2.1) -- +(0,4.2); % Poolstelle 2
\node[cross out, draw=red, thick] at (3.4,0) {};



% Nullstellen
\node[rounded rectangle, draw=blue, thick] at(0,0) {};
\node[rounded rectangle, draw=blue, thick] at(2.25,0) {};

\end{tikzpicture}\\
	\flushleft
	\vspace{-5pt}
	Als erstes wird das Verhalten des RET bei DC und bei sehr grossen Frequenzen analysiert. Für DC leiten beide Induktivitäten also ist dort eine NS. Für hohe Frequenzen leiten Beide Kapazitäten also ist dort ebenfalls NS.\\
	Da es bei dieser Schaltung um ein minimales RET handelt und aus vier Bauteilen besteht, hat der Reaktanzverlauf vier Pol/Null stellen. Diese treten abwechselnd auf.
\end{karte}

\begin{karte}{Ist folgendes RET ein minimales Eintor?\\[5pt]
		%Autor: Simon Walker
%Version: 1.0
%Datum: 15.04.2020
%Lizenz: CC BY-NC-SA

\begin{circuitikz}
	%\HelpCords{-2}{2}{5}{-5}
	
	\draw
	%Knoten
	(3, 0) coordinate (n1)
	(3, -2) coordinate (n2)
	($(n1)+(0.7,0)$) coordinate (n11)
	($(n2)+(0.7,0)$) coordinate (n22)
	(0.7, 0) coordinate (n3)
	(0.7, -2) coordinate (n4)
	(0, 0) coordinate (n5)
	(0, -2) coordinate (n6)
	
	
	%Nur zu Hilfszwecken
%	(n1) node[above] {$n1$}
%	(n2) node[above] {$n2$}
%	(n11) node[above] {$n11$}
%	(n22) node[above] {$n22$}
%	(n3) node[above] {$n3$}
%	(n4) node[above] {$n4$}
%	(n5) node[above] {$n5$}
%	(n6) node[above] {$n6$}
	
	(n5) to[short,o-] (n3)
	(n3) to[american inductor, *-*] (n4)
	(n3) to [american inductor] (n1)
	(n1) to[C, *-*] (n2)
	(n1) -- (n11) to[american inductor] (n22) -- (n2)
	(n2) to[short, -o] (n6)
	;
\end{circuitikz}}
	Nein! Denn es hat ein Ring aus Induktivitäten. In diesem Ring kann ein Strom fliesen. Dieser Strom ist von aussen nicht sichtbar. Eine Induktivität kann ersatzlos gestrichen werden, wenn dementsprechend die anderen Induktivitäten angepasst werden. Danach ist von aussen kein Unterschied feststellbar.\\
	\begin{center}
		%Autor: Simon Walker
%Version: 1.0
%Datum: 15.04.2020
%Lizenz: CC BY-NC-SA

\begin{circuitikz}
	%\HelpCords{-2}{2}{5}{-5}
	
	\draw
	%Knoten
	(3, 0) coordinate (n1)
	(3, -2) coordinate (n2)
	($(n1)+(0.7,0)$) coordinate (n11)
	($(n2)+(0.7,0)$) coordinate (n22)
	(0.7, 0) coordinate (n3)
	(0.7, -2) coordinate (n4)
	(0, 0) coordinate (n5)
	(0, -2) coordinate (n6)
	

	
	%Nur zu Hilfszwecken
%	(n1) node[above] {$n1$}
%	(n2) node[above] {$n2$}
%	(n11) node[above] {$n11$}
%	(n22) node[above] {$n22$}
%	(n3) node[above] {$n3$}
%	(n4) node[above] {$n4$}
%	(n5) node[above] {$n5$}
%	(n6) node[above] {$n6$}
	
	(n5) to[short,o-] (n3)
	(n3) to[american inductor, *-*] (n4)
	(n3) to [american inductor] (n1)
	(n1) to[C, *-*] (n2)
	(n1) -- (n11) to[american inductor] (n22) -- (n2)
	(n2) to[short, -o] (n6)
	;
	
	\draw[very thick, blue, rounded corners=5mm]
	(n3) -- (n11) -- (n22) -- (n4) -- cycle;
\end{circuitikz}
	\end{center}
\end{karte}

\begin{karte}{Welche Eigenschaften eines RET sind in der Praxis am unwahrscheinlichsten?}
	Es gibt zwei Probleme:
	\begin{compactitem}
		\item Impedanz wird nie 0 erreichen
		\item Die Impedanz kann nicht undendlich gross werden.
	\end{compactitem}
	Allerdings wird die Impedanz sich nicht komplett anders verhalten als die Reaktanz.
	%Autor: Simon Walker
%Version: 1.0
%Datum: 18.04.2020
%Lizenz: CC BY-NC-SA

\begin{tikzpicture}[smooth, xscale=0.7, yscale=0.7]
% Achsen
\draw[->, thick] (-0.2,0) -- +(5.2,0) node[right] {$\omega$}; % Horizontal
\draw[->, thick] (0,-2) -- +(0,4) node[above] {$\color{red}X \color{black}, \color{green} \left|Z\right|$}; % Vertikal

% Plots
\draw[color=red, thick] plot[domain=0.01:0.9] (\x*1.5,{tan((\x *1.2) r )}); % Erster Tan
\draw[color=red, thick] plot[domain=1.3:3.15] (\x*1.5,{tan((\x*1.2 -2.7) r)}); % Zweiter Tan
%Impedanz
%\draw[color=green, thick] plot[domain=0.01:0.9] (\x*1.5,{0.3+0.6*tan((\x *1.2) r )}); % Erster Tan

\draw [green, thick] plot [smooth] coordinates{
(0, 0.3) (0.5, 0.5) (1, 1) 
(1.5, 1.8)  (2, 1) (2.5, 0.5)
(3, 0.3) (3.5, 0.3) (4, 0.5) 
(4.5, 1) (5, 1.6) (5.5, 1.8)};


\draw [blue, thick] (3.34, 0) circle[radius=0.25];
\draw [blue, thick] (0, 0) circle[radius=0.25];
\draw [brown, thick] (1.35, 1.85) circle[radius=0.25];
\draw [brown, thick] (1.95, -2.15) circle[radius=0.25];
\draw [brown, thick] (4.73, 1.9) circle[radius=0.25];

\node [below] at (3.8, -0.2) {$\scriptstyle \color{blue} \left|Z\right| \ne 0 \Omega$};
\node at (3, 2) {$\scriptstyle \color{brown} \left|Z\right| \not\rightarrow \infty$};

\end{tikzpicture}

\end{karte}



	\kommentar{Leitungen}

\begin{karte}{Was ist eine Leitung?}
	Für uns ist eine Leitung eine Wellenführung.\\
	\divTwo{
		\textbf{Ungeführte Welle:}
		%Autor: Simon Walker
%Version: 1.0
%Datum: 24.04.2020
%Lizenz: CC BY-NC-SA

\begin{tikzpicture}[smooth, xscale=0.5, yscale=0.5]
	%\HelpCords{0}{0}{5}{4}
	
	\foreach \x in {0.6, 1.2, ...,4}{
		\draw[green] (\x, 0) arc (0:45:\x);	
	}
	
	\draw[<->, thick] (3, 3) -- (0, 0) -- (4.5, 0) node[right] {$\vec{r}$};
	\node[above right] at (3, 3) {$\vec{r}$};
\end{tikzpicture}
\\
		\scalebox{.7}{%Autor: Simon Walker
%Version: 1.0
%Datum: 24.04.2020
%Lizenz: CC BY-NC-SA

\begin{tikzpicture}[x={(0,0.8cm)}, y={(-0.55cm,-0.55cm)}, z={(1.2cm, -0.15cm)}]
		
	\def\cycles{1.75} %Anz Schwingungen
	\def\lenght{3}%länge in z richtung
	
	
	\draw [->, thick] (0,0,0)  -- (1.5,0,0) node[above] {\Large$x$};
	\draw [->, thick] (0,0,0) -- (0,1.2,0) node[below left] {\Large$y$};
	\draw [->, thick] (0,0,0) -- (0,0,\lenght+0.5) node[right] {\Large$z$};
	
	\draw[thick, green] %E-Feld Plot
	plot[domain=0:\lenght, samples=200] 
	({1.1*cos(deg(2*\cycles*pi*\x/\lenght))}, 0, \x);
	
	\draw[thick, blue] %M-Feld Plot
	plot[domain=0:\lenght, samples=200] 
	(0, {0.7*cos(deg(2*\cycles*pi*\x/\lenght))}, \x);
	
	\fill[opacity=.1,green!80] %E-Feld füllung
	(0,0,0) --
	plot[domain=0:\lenght, samples=200] 
	({1.1*cos(deg(2*\cycles*pi*\x/\lenght))}, 0, \x)  --
	(0,0,0);
	
	\fill[opacity=.1,blue!80] %M-Feld füllung
	(0,0,0) --
	plot[domain=0:\lenght, samples=200] 
	(0, {0.7*cos(deg(2*\cycles*pi*\x/\lenght))}, \x)  --
	(0,0,0);
	
	\node[green] at (1.3,0,0.3) {\Large$E$}; %Beschriftungen
	\node[blue] at (0,1.1,0.3) {\Large$B$};
	
\end{tikzpicture}
}
		\centering
		$\vec{k} \cdot \vec{r}$
		
	}{
		\textbf{Geführte Welle:}\\[5pt]
		%Autor: Simon Walker
%Version: 1.0
%Datum: 24.04.2020
%Lizenz: CC BY-NC-SA

\begin{tikzpicture}[smooth, xscale=0.4, yscale=0.4]
%	\HelpCords{0}{-1}{9}{1}

	\draw[thick] (-0.5, 0) -- (1, 0);
	\draw[thick] (8, 0) -- (9.5, 0);
	
	\draw[thick] (1, 0) ellipse circle [x radius=0.6, y radius=1];
	
	\draw[thick, fill=white] (8.2, 0) ellipse circle [x radius=0.6, y radius=1];
	\fill[white] (7,-1) rectangle (8.2,1);
	
	\foreach \x in {1.8, 2.4, ...,8}{
		\draw[green] (\x, -1) arc (-30:30:2);	
	}
	
	\draw[thick] (1, 1) -- (8.2, 1);
	\draw[thick] (1, -1) -- (8.2, -1);
	
	\foreach \x in {2.5, 4.25, ..., 8}{
		\draw[thick, blue] (\x, 0) ellipse circle [x radius=0.75, y radius=0.2];
	}

	\draw[thick, ->] (1, -1.8) -- (9, -1.8) node[right] {$z$};
\end{tikzpicture}

		\begin{center}
			$k\cdot z = \beta z \quad \Rightarrow \quad k=\beta$
		\end{center}
		Im Beispiel ist ein Koaxialleiter zu sehen. Das $B$-Feld ist Ortogonal zum $E$-Feld. Die gesamte Anordnung ist Symetrisch.
	}
\end{karte}

\begin{karte}{Wie Sieht das Schema des Leitungsmodells aus?}
	Das Leitungsmodell bezieht sich auf einen kurzen Leitungsausschnitt. Dadurch erhalten wir längenunabhängige Kenngrössen die sogenanten Leitungsbeläge.\\
	%\divTwo{
	%\begin{figure}[H]
	%	\centering
		\scalebox{.49}{%Autor: Simon Walker
%Version: 1.0
%Datum: 04.06.2020
%Lizenz: CC BY-NC-SA

\begin{circuitikz}
	%\HelpCords{0}{0}{8}{-4}
	
	
	\draw
	%Knoten
	(0, 0) coordinate (n1)
	
	($(n1)+(0.1,0)$) coordinate(n11)
	
	(1.5, 0) coordinate (n2)
	(3.5, 0) coordinate (n3)
	(5.5, 0) coordinate (n4)
	
	($(n4)+(0,-0.1)$) coordinate(n41)
	
	(8, 0) coordinate (n5)
	
	($(n5)+(-0.1,0)$) coordinate(n51)
	
	
	($(n4)+(0, -1)$) coordinate (n6)
	($(n6)+(0, +0.1)$) coordinate (n60)
	($(n6)+(-0.8, 0)$) coordinate (n61)
	($(n6)+(0.8, 0)$) coordinate (n62)
	
	($(n6)+(0, -2)$) coordinate (n7)
	($(n61)+(0, -2)$) coordinate (n71)
	($(n62)+(0, -2)$) coordinate (n72)
		
	($(n7)+(0, -0.5)$) coordinate (n9)
	
	(8, -3.5) coordinate (n8)
	(0, -3.5) coordinate (n10)
	
	
	
	%Nur zu Hilfszwecken
%	(n1) node[above] {$n1$}
%	(n2) node[above] {$n2$}
%	(n3) node[above] {$n3$}
%	(n4) node[above] {$n4$}
%	(n5) node[above] {$n5$}
%	(n6) node[above] {$n6$}
%	(n7) node[above] {$n7$}
%	(n8) node[above] {$n8$}
%	(n9) node[above] {$n9$}
%	(n10) node[above] {$n10$}
%	
%	(n11) node[below] {$n11$}
%	(n41) node[below] {$n41$}
%	(n51) node[below] {$n51$}
%	(n60) node[below] {$n60$}
%	(n61) node[above] {$n61$}
%	(n62) node[above] {$n62$}
%	(n71) node[above] {$n71$}
%	(n72) node[above] {$n72$}
	
	(n1) to [short,o-] (n11)
	(n11) to[short,color=red,i=${\color{red} i(z,t)}$] (n2)
	(n2) to[R=$R'dz$] (n3)
	(n3) to [american inductor=$L'dz$] (n4)
	(n4) to[short, color=red, i=${\color{red} i(z+dz,t)}$] (n51)
	(n51) to[short,-o] (n5)
	
	(n4) to[short, *-] (n41)
	(n41) to[short, color=red, i=${\color{red} \Delta i(z,t)}$] (n60)
	(n60) to[short, -*] (n6)
	(n6) to[short] (n61)
	(n6) to[short] (n62)
	(n61) to[R,l_=$G'dz$] (n71)
	(n62) to[C=$C'dz$] (n72)	
	(n7) to[short] (n71)
	(n7) to[short] (n72)
	(n7) to[short, *-*] (n9)
	
	(n10) to[short, o-o] (n8)
	;
	
	\draw[|<->|] ($(n10)+(0,-0.7)$) --
	node[above] {$dz$}
	($(n8)+(0,-0.7)$);
	
	\draw[blue,->] (0.3, 0.6) to[out=10,in=170] (7.3, 0.6);
	\node[above, blue] at (3.75, 0.9) {$\Delta u(z)$};
	
	
	\draw[blue,->] ($(n1)+(0, -0.3)$) --
	node[left, blue] {$u(z,t)$}
	($(n10)+(0, +0.3)$);
	
	\draw[blue,->] ($(n5)+(0, -0.3)$) --
	node[right, blue] {$u(z+dz,t)$}
	($(n8)+(0, +0.3)$);
	
\end{circuitikz}}\\[5pt]
	%\end{figure}
	%}{
	\tiny
	\renewcommand\arraystretch{1.7}
	\begin{tabular}{cl}
		$R'=\frac{dR}{dz}$ & Widerstandsbelag in $\Omega/m$\\
		$L'=\frac{dL}{dz}$ & Induktivitätsbelag in $H/m$\\
		$G'=\frac{dG}{dz}$ & Isolationsbelag in $S/m$\\
		$C'=\frac{dC}{dz}$ & Kapazitätsbelag in $F/m$\\
	\end{tabular}
	\normalsize
	%}


\end{karte}

\begin{karte}{Was ist eine TEM-Welle?}
	TEM steht für transversalelektomagnetisch. Eine TEM-Welle hat nur Komponenten, welche senkrecht zur Ausbreitungsrichtung sind. Im Bild ist $z$ die Ausbreitungsrichtung der Welle und steht senkrecht zu $\vec{E}$ und $\vec{H}$. \\[10pt]
	%Autor: Simon Walker
%Version: 1.0
%Datum: 28.04.2020
%Lizenz: CC BY-NC-SA

\begin{tikzpicture}[x={(0, 0.8cm)}, y={(-0.5cm,-0.5cm)}, z={(0.8cm, 0)}]

\def\l{7} %länge der z-Achse
\def\s{1.2} %Quadratgrösse (eine Seite ist s*2)
\def\p{5} %Positoon des Quadrates

\draw[domain=90:180,smooth,variable=\x] plot ({sin(\x)*0.5},{0}, {1.5+cos(\x)*0.5});
\draw[domain=0:90,smooth,variable=\x] plot ({0},{sin(\x)*0.5}, {1.5+cos(\x)*0.5});
\draw[fill=black] (0.2, 0, 1.3) circle[radius=0.03];
\draw[fill=black] (0, 0.2, 1.7) circle[radius=0.03];
\draw[very thick, green, ->] (0, 0, 1.5) -- (1, 0, 1.5) node[above, green] {$\vec{E}$};
\draw[very thick, blue, ->] (0, 0, 1.5) -- (0, 1, 1.5) node[below left, blue] {$\vec{H}$};

\draw[thick] (0, 0, 0) -- (0, 0, \p); %Beginn der Z-Achse

\draw[fill=gray!10] (-\s, -\s, \p) -- (-\s, \s, \p) -- (\s, \s, \p) -- (\s, -\s, \p) -- cycle;
\draw[->,domain=-75:195,smooth,variable=\x, blue, thick] plot ({cos(\x)*0.5},{sin(\x)*0.5}, {\p});
\node[blue] at (-0.6, 0.5, \p) {$B(I)$};
\draw[->, green, thick] (0, 0, \p) -- (1, 0, \p) node[right, green] {$U$};
\draw[thick, ->] (0, 0, \p) -- (0, 0, \l) node[right] {$z$}; %Ende der Z-Achse

\end{tikzpicture}

\end{karte}

\begin{karte}{Was bedeutet der Begriff Wellenimpedanz?}
	Die Wellenimpedanz ist das Verhältnis zwischen Spannung und Strom der Welle. Achtung: Dies ist nicht gleich der gemessenen Spannung und Strom an einem gewissen Punkt $z$ in der Leitung!\\[5pt]
	%Autor: Simon Walker
%Version: 1.0
%Datum: 30.04.2020
%Lizenz: CC BY-NC-SA

\begin{tikzpicture}[smooth, xscale=0.6, yscale=0.6]
%	\HelpCords{0}{-1}{9}{1}

	\draw[thick] (-1, 0) -- (1, 0);
	\draw[thick] (8, 0) -- (10.5, 0);
	
	\draw[thick] (1, 0) ellipse circle [x radius=0.6, y radius=1];
	
	\draw[thick, fill=white] (8.2, 0) ellipse circle [x radius=0.6, y radius=1];
	\fill[white] (7,-1) rectangle (8.2,1);
	
	
	\draw[thick] (1, 1) -- (8.2, 1);
	\draw[thick] (1, -1) -- (8.2, -1);
	
	%Pfeile
	\draw [->, blue] (2, 0) -- ++(0.3,0) 
	arc [radius=0.2, start angle=180, end angle = 0] 
	arc [radius=0.2, start angle=-180, end angle = 0]
	arc [radius=0.2, start angle=180, end angle = 0] -- ++(0.3,0);
	\node [blue] at (3, 0.6) {$U_+$};
	\node [blue] at (3, -0.6) {$I_+$};
	
	\draw [->, blue] (8, 0) -- ++(-0.3,0) 
	arc [radius=0.2, start angle=0, end angle = 180] 
	arc [radius=0.2, start angle=0, end angle = -180]
	arc [radius=0.2, start angle=0, end angle = 180] -- ++(-0.3,0);
	\node [blue] at (7, 0.6) {$U_-$};
	\node [blue] at (7, -0.6) {$I_-$};

	\draw[thick, ->] (1, 1.7) -- (9, 1.7) node[right] {$z$};
	
	\draw[very thick, ->] (2, -1.3) -- ++(0, -1.2) -- ++(1.2, 0) node [right]
	{$Z_0 = \color{blue} \dfrac{U_+}{I_+} \color{black} = 
	\color{blue} \dfrac{U_-}{I_-} \color{red} \ne \dfrac{U(z)}{I(z)}$}; 
	\node[right] at (3.2, {-2.5-1.3}) 
	{$\color{white} Z_0 \color{black} = \sqrt{\frac{R'+j \omega L'}{G' +j \omega C'}}= Z(\omega) \text{\tiny{\quad gemäss Leitungsmodel}}$} ;
	
\end{tikzpicture}
 
\end{karte}

\begin{karte}{Was bezeichnet man als Ausbreitungsgeschwindigkeit?}
	Für uns ist meistens mit der Ausbreitungsgeschwindigkeit die Phasengeschwindigkeit gemeint.
	\begin{equation*}
		v = \frac{dz}{dt} = v_{ph} 
	\end{equation*}
	Am einfachsten wird geschaut wie sich der Peak der Welle fortbewegt.
	\vspace{-10pt}
	\begin{figure}[h]
		\centering
		%Autor: Simon Walker
%Version: 1.0
%Datum: 30.04.2020
%Lizenz: CC BY-NC-SA

\begin{tikzpicture}[xscale=0.7, yscale=0.7]
	%\HelpCords{0}{-1}{9}{1}
	
	\def\l{8} %Länge der Schwingungen
	\def\o{0.9} %Offset (Positiv bedeutet Rot links von blau)
	\def\f{0.3} %Frequenz der Schwingunen
	
	\tikzmath{
		\p1 = 450/(360*\f); %Peek1
		\p2 = \p1 - \o; %Peek2
		\pl = (\p1+\p2)/2;
	}
	
	\draw[blue] plot[domain=0:\l, samples=200] 
	(\x, {sin(360*\f*\x});
	\draw[red] plot[domain=0+\o:\l+\o, samples=200] 
	(\x-\o, {sin(360*\f*\x)});
	
	%\draw[green, *<-*] (\p2, 1) -- (\p1, 1);
	\draw [thick] (\p1, 1) -- (\p1, 1.5);
	\draw [thick](\p2, 1) -- (\p2, 1.5);
	\draw [->, thick] (\p1, 1.25) -- (\p2, 1.25);
	\node[above] at (\pl, 1.25) {$dz$};
	
	\draw[fill=green, green] (\p1, 1) circle [radius=0.08];
	\draw[fill=green, green] (\p2, 1) circle [radius=0.08];
	
\end{tikzpicture}

	\end{figure}
\end{karte}

\begin{karte}{Was ist die Ausbreitungskonstante?}
	Die Ausbreitungskonstante $\gamma$ ist eine komplexe Grösse und hat die Einheit $Np/m + jrad/m$. $Np/m$ ist (Neper/Meter).\\[7pt]
	\begin{minipage}[c]{0.38\textwidth}
		\begin{equation*}
		\gamma = \alpha + j \beta
		\end{equation*}
	\end{minipage}
	\begin{minipage}[t]{0.68\textwidth}
		\begin{tabular}{ccl}
			$\gamma$ & => & Ausbreitungskonstante\\
			$\alpha$ & => & Dämpfungskonstante\\
			$\beta$  & => & Phasenkonstante
		\end{tabular}
	\end{minipage}\\[7pt]
	
	Eine Welle kann mittels der Ausbreitungskonstante definiert werden:
	\begin{equation*}
		U_+ e^{-\gamma z}
	\end{equation*}
	Wobei $U_+ e^{-\gamma z}$ eine Welle in die Positive $z$ Richtung ist mit der komplexen Amplitude $U_+=|U_+| e^{j \varphi_+}$
\end{karte}

\begin{karte}{Wie entstehen durch Leitungen Verzerrungen und wie können diese verhindert werden?}
	Eine Möglichkeit ist die \textbf{Dispersion}. Dies ist dann der Fall wenn $\beta(\omega)$, $\gamma(\omega)$, $Z(\omega)$, $v(\omega)$ Frequenzabhänig sind.\\
	Eine andere Möglichkeit für Verzerrungen sind Reflexionen. Die Reflexionen treten immer Zeitverzögert auf ($\tau$).\\
	\scalebox{.8}{%Autor: Simon Walker
%Version: 1.0
%Datum: 15.04.2020
%Lizenz: CC BY-NC-SA

\begin{circuitikz}
	%\HelpCords{0}{0}{8}{3}
	
	
	\draw
	%Knoten
	(0, 0) coordinate (n1)
	(0, 3) coordinate (n2)
	(2, 3) coordinate (n3)
	(6, 3) coordinate (n4)
	(7, 3) coordinate (n5)
	(7, 0) coordinate (n6)	
	
	
	%Nur zu Hilfszwecken
%	(n1) node[above] {$n1$}
%	(n2) node[above] {$n2$}
%	(n3) node[above] {$n3$}
%	(n4) node[above] {$n4$}
%	(n5) node[above] {$n5$}
%	(n6) node[above] {$n6$}
	
	(n2) to[sV] (n1) node[rground]{}
	(n2) to[R] (n3)
	(n4) to[short] (n5) to[R=${Z_a}$] (n6) node[rground]{}
	;
	
	%Leitung
	\draw
	(n3) to[short] ($(n3)+(0.5, 0)$)
	(n4) to[short] ($(n4)+(-2, 0)$);
	
	\draw[fill=white] ($(n4)+(-0.5, 0)$) ellipse circle [x radius=0.3, y radius=0.5];
	\fill[white] ($(n3)+(0.5, -0.5)$) rectangle ($(n4)+(-0.5, 0.5)$);
	\draw ($(n3)+(0.5, 0)$) ellipse circle [x radius=0.3, y radius=0.5];
	\draw ($(n3)+(0.5, -0.5)$) -- ($(n4)+(-0.5, -0.5)$);
	\draw ($(n3)+(0.5, +0.5)$) -- ($(n4)+(-0.5, +0.5)$);
	
	\draw[red,|->] ($(n3)+(0,0.8)$) -- 
	node[above, red] {$t=\tau$}
	($(n4)+(0,0.8)$);
	
	\path ($(n3)+(0,-0.5)$) -- 
	node[below] {$Z_0$}
	($(n4)+(0,-0.5)$);
	
	\draw[dashed] ($(n4)+(0,1)$) -- ($(n4)+(0,-3.5)$);
	
	\draw[<-, very thick] ($(n4)+(0,-1.5)$) -| ++(-0.8,-0.5) node[below] {$\Gamma$};
	

\end{circuitikz}}
\end{karte}

	\kommentar{Zweitore}

\begin{karte}{Worum handelt sich bei einem Zweitor?}
	Ein Zweitor ist ein Vierpol bei welchem immer zwei als Paar auftreten. Die Spannungen und Ströme sind wie im Schema eingezeichnet definiert. Es muss immer auf das Vorzeichen geachtet werden. Ansonsten funktionieren die gesamten Rechenregeln nicht.
	
	\centering{\scalebox{.9}{%Autor: Simon Walker
%Version: 1.0
%Datum: 25.06.2020
%Lizenz: CC BY-NC-SA


\begin{tikzpicture}

%Symbol
\draw[thick] (-1, -1.3) rectangle (1, 1.3);
\draw[thick] (-1,0.8) -- (-2,0.8);
\draw[thick, fill=white] (-2,0.8) circle (0.1);
\draw[thick] (-1,-0.8) -- (-2,-0.8);
\draw[thick, fill=white] (-2,-0.8) circle (0.1);

\draw[thick] (1,0.8) -- (2,0.8);
\draw[thick, fill=white] (2,0.8) circle (0.1);
\draw[thick] (1,-0.8) -- (2,-0.8);
\draw[thick, fill=white] (2,-0.8) circle (0.1);

%Strompfeile
\draw[red, thick] (-1.5, 0.8) node[above, red] {$i_1$} ++(-0.15, 0.15) -- ++(0.15, -0.15) -- ++(-0.15, -0.15);
\draw[red, thick] (-1.5, -0.8) node[above, red] {$i_1$} ++(0.15, 0.15) -- ++(-0.15, -0.15) -- ++(0.15, -0.15);

\draw[red, thick] (1.5, 0.8) node[above, red] {$i_2$} ++(0.15, 0.15) -- ++(-0.15, -0.15) -- ++(0.15, -0.15);
\draw[red, thick] (1.5, -0.8) node[above, red] {$i_2$} ++(-0.15, 0.15) -- ++(0.15, -0.15) -- ++(-0.15, -0.15);

%Spannungspfeile
\draw[blue, ->, thick, out=-110, in=110] (-2.1, 0.7) to (-2.1, -0.7);
\node[left, blue] at (-2.3,0){$u_1$};

\draw[blue, ->, thick, out=-60, in=60] (2.1, 0.7) to (2.1, -0.7);
\node[right, blue] at (2.3,0){$u_2$};


\end{tikzpicture}
}}	
\end{karte}

\begin{karte}{Was versteht man im Bezug auf Zweitore unter Passivität, Reziprozität und Symmetrie?}
	\textbf{Passivität:} Die Leistung die das Zweitor aufnimmt ist grösser als 0 ($P\ge 0$)\\
	\textbf{Reziprozität:} Ist die Symmetrie bezüglich dem Ausgang. Die Ströme durch die Amperemeter (blau umkreist) müssen gleich sein.\\
	\textbf{Symmetrie:} Die Ströme durch die Quelle (grün umkreist) müssen ebenfalls gleich sein.
	
	\centering{\scalebox{.7}{%Autor: Simon Walker
%Version: 1.0
%Datum: 25.06.2020
%Lizenz: CC BY-NC-SA


\begin{circuitikz}
%Symbol1
\draw[thick] (-1, -0.8) rectangle (1, 0.8);
\draw (-1, 0.5) to[short, -o] (-2, 0.5);
%\draw[thick] (-1,0.5) -- (-2,0.5);
%\draw[thick, fill=white] (-2,0.5) circle (0.1);
\draw (-1, -0.5) to[short, -o] (-2, -0.5);
%\draw[thick] (-1,-0.5) -- (-2,-0.5);
%\draw[thick, fill=white] (-2,-0.5) circle (0.1);

\draw[thick] (1,0.5) -- (2,0.5);
\draw[thick, fill=white] (2,0.5) circle (0.1);
\draw[thick] (1,-0.5) -- (2,-0.5);
\draw[thick, fill=white] (2,-0.5) circle (0.1);

%Strompfeile
\draw[red, thick] (-1.5, 0.5) node[above, red] {$i_1$} ++(-0.15, 0.15) -- ++(0.15, -0.15) -- ++(-0.15, -0.15);

\draw[red, thick] (1.5, 0.5) node[above, red] {$i_2$} ++(0.15, 0.15) -- ++(-0.15, -0.15) -- ++(0.15, -0.15);

%Quelle
\draw (-2,0.5) to[short] (-2,1) to[short] (-2.5, 1)	to[american voltage source] (-2.5 ,-1) to[short] (-2,-1) to[short] (-2, -0.5);

%Amperemeter
\draw (2,0.5) to[short] (2,1) to[short] (2.5, 1) to node[draw,circle,fill=white] {A} (2.5 ,-1) to[short] (2,-1) to[short] (2, -0.5);

%Einkreisen
\draw[green, very thick] (-1.5, 0.5) circle (0.5);
\draw[blue, very thick] (1.5, 0.5) circle (0.5);


\begin{scope}[shift={(7,0)}]
	%Symbol2
	\draw[thick] (-1, -0.8) rectangle (1, 0.8);
	\draw (-1, 0.5) to[short, -o] (-2, 0.5);
	%\draw[thick] (-1,0.5) -- (-2,0.5);
	%\draw[thick, fill=white] (-2,0.5) circle (0.1);
	\draw (-1, -0.5) to[short, -o] (-2, -0.5);
	%\draw[thick] (-1,-0.5) -- (-2,-0.5);
	%\draw[thick, fill=white] (-2,-0.5) circle (0.1);
	
	\draw[thick] (1,0.5) -- (2,0.5);
	\draw[thick, fill=white] (2,0.5) circle (0.1);
	\draw[thick] (1,-0.5) -- (2,-0.5);
	\draw[thick, fill=white] (2,-0.5) circle (0.1);
	
	%Strompfeile
	\draw[red, thick] (-1.5, 0.5) node[above, red] {$i_1$} ++(-0.15, 0.15) -- ++(0.15, -0.15) -- ++(-0.15, -0.15);
	
	\draw[red, thick] (1.5, 0.5) node[above, red] {$i_2$} ++(0.15, 0.15) -- ++(-0.15, -0.15) -- ++(0.15, -0.15);
	
	%Quelle
	\draw (-2,0.5) to[short] (-2,1) to[short] (-2.5, 1)	to node[draw,circle,fill=white] {A} (-2.5 ,-1) to[short] (-2,-1) to[short] (-2, -0.5);
	
	%Amperemeter
	\draw (2,0.5) to[short] (2,1) to[short] (2.5, 1) to[american voltage source] (2.5 ,-1) to[short] (2,-1) to[short] (2, -0.5);
	
	%Einkreisen
	\draw[green, very thick] (1.5, 0.5) circle (0.5);
	\draw[blue, very thick] (-1.5, 0.5) circle (0.5);
	
\end{scope}


\end{circuitikz}
}}
\end{karte}

	\kommentar{EMV}

\begin{karte}{Was bedeutet EMV?}
	EMV steht für Elektomagnetische Verträglichkeit. Das bedeutet im Wesentlichen, dass das Gerät in einem Umfeld mit gewissen Störungen störungsfrei funktioniert aber auch dass das Gerät das Umfeld nicht zu stark stört. Ein wenig Stören ist erlaubt. Dies ist von der Kategorie abhängig.
\end{karte}

\begin{karte}{Welche drei Perspektiven gibt es im Zusammenhang mit EMV?}
	Die drei Perspektiven sind \textbf{SI}, \textbf{PI} und \textbf{EMC}.\\[5pt]
	\textbf{SI} steht für Signal Integrity und behandlet vorallem die Frage, Wie gut sind die Signale intern. Werden alle Anforderungen bezüglich den standardisierten Signalen eingehalten?\\[5pt]
	\textbf{PI} steht für Power Integrity. Das ist die Koexistens Intern. Die Verschieden Teile eines System dürfen sich nicht zu viel stören. Ein bisschen Störung ist allerdings nicht zu vermeiden. Die Störungen werden primär durch die gesamte Leistungs-/Energieversorgung von einem Teil auf den Anderen übertragen.\\[5pt]
	\textbf{EMC} steht für Electromagnetic Compatibility. Das ist die Koexistenz gegen Aussen.
\end{karte}

\begin{karte}{Was beinhaltet das Grundmodell der Störungsbeeinflussung?}
	Das Grundmodell basiert auf drei Komponenten. Es gibt Störquellen, Kopplungspfade und Störsenken.\\
	%\scalebox{.7}{\input{tikz/ModellStörung.tex}}\\
	\input{tikz/ModellStörung.tex}\\[5pt]
	Bei der Senke kann eigentlich nichts unternommen werden um Störungen zu vermeiden. Bei der Quelle kann man allfällige Störungen versuchen zu verhindern. Beispielsweise indem steile Flanken vermieden werden. Am meisten Einfluss haben wir allerdings auf die Kopplung. Wenn wir die Kopplung reduzieren wird die Störbeeinflussung ebenfalls reduziert.
\end{karte}

\begin{karte}{Welche Kopplungsarten von Störungen werden unterschieden?}
	Es gibt die kapazitive, die induktive, die galvanische und die Strahlen Kopplung.
\end{karte}

	
	%Zusätzliche Karten Version 2
	\setcounter{CardID}{100}
	\kommentar{Recap ELT3}

\begin{karte}{Was ist eine Reaktanz}
	Die Reaktanz ($X$) ist der Blindwiderstand. Somit ist er der Imaginärteil der Impedanz ($Z$). 
	Die Reaktanz lässt sich folgendermassen berechnen:\\[10pt]
	\begin{minipage}[t]{0.48\textwidth}
		Für Kapazitäten:\\
		$X = - \dfrac{1}{\omega C}$
	\end{minipage}
	\begin{minipage}[t]{0.48\textwidth}
		Für Induktivitäten:\\
		$X = \omega L$
	\end{minipage}\\[10pt]
	Da die Impedanz $Z = jX$ ist folgt:\\[10pt]
	\begin{minipage}[t]{0.48\textwidth}
		Für Kapazitäten:\\
		$Z = jX = - \dfrac{j}{\omega C} = \dfrac{1}{j \omega C}$
	\end{minipage}
	\begin{minipage}[t]{0.48\textwidth}
		Für Induktivitäten:\\
		$Z = jX = j \omega L$
	\end{minipage}
\end{karte}

	\kommentar{Frequenzabhängigkeit}


	\kommentar{Schwingkreise}

\begin{karte}{Was ist ein Schwingkreis und welche Betriebsmodus gibt es?}
	Ein Schwingkreis ist:\\
	\begin{itemize}
		\item ein Netzwerk aus Ohmischen und Kapazitive Bauteile
		\item ein resonanzfähiges Netzwerk
		\item Kann parallel oder seriell geschaltet werden
	\end{itemize}
	Es gibt 2 Betriebsmodus:
	\begin{itemize}
		\item Freie Schwingung 
		\item Erzwungene Schwingung (Durch Quelle angeregt)
	\end{itemize}
\end{karte}

\begin{karte}{Welche Frequenzen gibt es in einem Schwingkreis?}
	Es gibt drei verschiedene Frequenzen in einem Schwingkreis:
	\begin{itemize}
		\item Eigenfrequenz $\omega_0$\\
		(Resonanzfrequenz des verlustlosen Schwingkreis)
		\item Resonanzfrequenz $\omega_r$\\
		($Im(Z) = 0$, Reine Reelle Impedanzen und Admitanzen)
		\item Extremalfrequenz $\omega_m$\\
		(Frequenz bei welcher die Impedanz maximal oder minimal ist)
	\end{itemize}
\end{karte}

\begin{karte}{Wie ist die Bandbreite eines Schwingkreises definiert?}
	\textbf{(absolute) Bandbreite}:
	\begin{equation*}
		B = f^+ - f^- = \dfrac{\omega^+ - \omega^-}{2 \pi}
	\end{equation*}
	Die Bandbreite ist der Frequenzbereich, in welchem die Impedanz bzw. Admittanz des Resonators sich nur um den Faktor $\sqrt{2}$ vom nächsten (lokalen) Minimum $\omega_m$ unterscheidet.\\
	Oder Formal ausgedrückt: \quad $|Z (\omega^{\pm})|=\sqrt{2}\cdot|Z(\omega_{m})|$\\
	\textbf{relative Bandbreite}:
	\begin{equation*}
	B_{rel} = \frac{f^+ - f^-}{f_m} = \frac{\omega^+ - \omega^-}{\omega_m} =\frac{1}{Q_s}
	\end{equation*}
	Wobei $Q_s$ die Schwingkreisgüte ist.
\end{karte}

\begin{karte}{Was bedeutet Resonanz?}
	\begin{itemize}
		\item Aus der Akustik:\\
		Verstärktes Mitschwingen eines schwingfähigen Systems
		\item Die Resonanzfrequenz $\omega_r$ ist die Frequenz bei welcher die elektrische und magnetische Energie gleich gross sind $W_C = W_L$.
	\end{itemize}
	Bei der Resonanzfrequenz gilt folglich:
	\begin{itemize}
		\item $|X_C| = |X_L|$
		\item $Im(Z) = 0$
		\item $\tan (\varphi) = 0 \quad \rightarrow \quad \varphi = 0$
	\end{itemize}
\end{karte}

\begin{karte}{Wie verhält sich ein echter Kondensator?}
	Ein Kondensator ist laut dem Ersatzschaltbild ein Serienschwingkreis und hat dementsprechend folgenden Verlauf:
	
	\begin{minipage}{0.48\textwidth}
		\scalebox{.7}{%Autor: Simon Walker
%Version: 1.0
%Datum: 19.04.2020
%Lizenz: CC BY-NC-SA

\begin{circuitikz}
	%	\HelpCords{-1}{-1}{7}{1}
	
	\draw
	%Knoten
	(0, 0) coordinate (n1)
	(2, 0) coordinate (n2)
	(4, 0) coordinate (n3)
	(6, 0) coordinate (n4)
	
	%Nur zu Hilfszwecken
	%	(n1) node[above] {$n1$}
	%	(n2) node[above] {$n2$}
	%	(n3) node[above] {$n3$}
	%	(n4) node[above] {$n4$}
	
	(n1) to[C=$C$, o-] (n2) to [R=$R(\omega)$] (n3) 
	to [american inductor=$L$, -o] (n4)
	;
\end{circuitikz}
}
	\end{minipage}
	\begin{minipage}{0.48\textwidth}
		%Autor: Simon Walker
%Version: 1.0
%Datum: 19.04.2020
%Lizenz: CC BY-NC-SA

\begin{tikzpicture}[smooth, xscale=0.55, yscale=0.55]
	% Achsen
	\draw[->, thick] (-0.2,0) -- (5.2,0) node[right] {$\scriptstyle log \, \omega$}; % Horizontal
	\draw[->, thick] (0,-0.5) -- (0,2.7) node[above] {$ \scriptstyle \color{red} \left|Z\right|$}; % Vertikal
	
	%Impedanzfunktion
	\draw [red, thick] plot [smooth] coordinates{
		(0.2, 2.5) (2.5, 0.3) (4.8, 2.5)
	};

	%Beschriftungen
	\draw [blue, dashed] (2.8, 0.3) -- (-0.2, 0.3)
	node[left] {$\scriptstyle ESR$};
	
	
\end{tikzpicture}

	\end{minipage}\\[5pt]
	Das Minimum der Impedanzfunktion ist rein reell und wird ESR (Equivalent Series Resistance) genant. Es enspricht dem Wert des Seriewiderstands bei der entsprechenden Frequenz.
\end{karte}

\begin{karte}{Wie verhält sich ein echten Widerstand?}
	Das Ersatzschaltbild eines Widerstands hat entweder eine Spule in Serie (für  kleine $R$) oder einen Kondensator Parallel (für grosse $R$).
	
	\begin{minipage}{0.48\textwidth}
		\scalebox{.85}{%Autor: Simon Walker
%Version: 1.0
%Datum: 19.04.2020
%Lizenz: CC BY-NC-SA

\begin{circuitikz}
%	\HelpCords{-1}{-3}{5}{1}
	
	\draw
	%Knoten
	(0, -2) coordinate (n11)
	(1, -2) coordinate (n12)
	(1, -2.75) coordinate (n13)
	(1, -1.25) coordinate (n14)
	(3, -2.75) coordinate (n15)
	(3, -1.25) coordinate (n16)
	(3, -2) coordinate (n17)
	(4, -2) coordinate (n18)
	
	(0, 0) coordinate (n21)
	(2, 0) coordinate (n22)
	(4, 0) coordinate (n23)
	
	
	%Nur zu Hilfszwecken
%	(n11) node[above] {$n11$}
%	(n12) node[above] {$n12$}
%	(n13) node[above] {$n13$}
%	(n14) node[above] {$n14$}
%	(n15) node[above] {$n15$}
%	(n16) node[above] {$n16$}
%	(n17) node[above] {$n17$}
%	(n18) node[above] {$n18$}
	
%	(n21) node[above] {$n21$}
%	(n22) node[above] {$n22$}
%	(n23) node[above] {$n23$}
	;
	%Serie L
	\draw[red]
	(n21) to[R=$R$, o-, color=red] (n22) 
	to [american inductor=$L$, -o, color=red] (n23)
	;
	
	
	%Parallel C
	\draw[blue]
	(n11) to[short, o-*, color=blue] (n12)
	(n13) to[short, color=blue] (n14)
	(n14) to[R=$R$, color=blue] (n16)
	(n13) to[C=$C$, color=blue] (n15)
	(n16) to[short, color=blue] (n15)
	(n17) to[short, *-o, color=blue] (n18)
	;
\end{circuitikz}
}
	\end{minipage}
	\begin{minipage}{0.48\textwidth}
		%Autor: Simon Walker
%Version: 1.0
%Datum: 19.04.2020
%Lizenz: CC BY-NC-SA

\begin{tikzpicture}[smooth, xscale=0.7, yscale=0.7]
	% Achsen
	\draw[->, thick] (-0.2,0) -- (5.2,0) node[right] {$\scriptstyle log \, \omega$}; % Horizontal
	\draw[->, thick] (0,-0.5) -- (0,2.7) node[above] {$ \scriptstyle \left|Z\right|$}; % Vertikal
	
	%Impedanzfunktion Serie (kleines R)
	\draw [red, thick, rounded corners=8mm] 
		(0, 0.5) -- (3.5, 0.5) -- (4.8, 2.5);
		
	%Impedanzfunktion Parallel (grosses R)
	\draw [blue, thick, rounded corners=8mm] 
	(0, 2.3) -- (3.5, 2.3) -- (4.8, 0.5);
	
\end{tikzpicture}

	\end{minipage}\\[5pt]
\end{karte}

\begin{karte}{Wie verhält sich eine echte Spule?}
	Das Ersatzschaltbild einer Spule ist ein Parallelschwingkreis. Deshalb hat der Impedanzverlauf folgende Eigenschaften.
	
	\begin{minipage}{0.48\textwidth}
		%Autor: Simon Walker
%Version: 1.0
%Datum: 19.04.2020
%Lizenz: CC BY-NC-SA

\begin{circuitikz}
%	\HelpCords{-1}{-1}{5}{1}
	
	\draw
	%Knoten
	(0, 0) coordinate (n1)
	(1, 0) coordinate (n2)
	(1, -0.75) coordinate (n3)
	(1, 0.75) coordinate (n4)
	(3, -0.75) coordinate (n5)
	(3, 0.75) coordinate (n6)
	(3, 0) coordinate (n7)
	(4, 0) coordinate (n8)
	
	
	%Nur zu Hilfszwecken
%	(n1) node[above] {$n1$}
%	(n2) node[above] {$n2$}
%	(n3) node[above] {$n3$}
%	(n4) node[above] {$n4$}
%	(n5) node[above] {$n5$}
%	(n6) node[above] {$n6$}
%	(n7) node[above] {$n7$}
%	(n8) node[above] {$n8$}
	
	;	


	\draw
	(n1) to[short, o-*] (n2)
	(n3) to[short] (n4)
	(n4) to[american inductor=$L$] (n6)
	(n3) to[C=$C$] (n5)
	(n6) to[short] (n5)
	(n7) to[short, *-o] (n8)
	;
\end{circuitikz}

	\end{minipage}
	\begin{minipage}{0.48\textwidth}
		%Autor: Simon Walker
%Version: 1.0
%Datum: 19.04.2020
%Lizenz: CC BY-NC-SA

\begin{tikzpicture}[smooth, xscale=0.7, yscale=0.7]
	% Achsen
	\draw[->, thick] (-0.2,0) -- (5.2,0) node[right] {$\scriptstyle log \, \omega$}; % Horizontal
	\draw[->, thick] (0,-0.5) -- (0,2.7) node[above] {$ \scriptstyle \left|Z\right|$}; % Vertikal
	
		%Impedanzfunktion
	\draw [red, thick, rounded corners=2mm] 
		(0.2, 0.5) -- (2.5, 2.5) -- (4.8, 0.5);
	
\end{tikzpicture}

	\end{minipage}\\[5pt]
\end{karte}

	\kommentar{Reaktanzeintore (RET)}

\begin{karte}{Was ist ein Reaktanzeintor?}
	Ein Reaktanzeintor hat zwei Anschlüsse, besteht nur aus Reaktanzen und hat somit keine Wirkwiderstände.
	\flushleft
	\begin{minipage}{0.32\textwidth}
		Beispiel:
		\scalebox{.8}{%Autor: Simon Walker
%Version: 1.0
%Datum: 15.04.2020
%Lizenz: CC BY-NC-SA

\begin{circuitikz}
	%\HelpCords{-2}{2}{5}{-5}
	
	\draw
	%Knoten
	(2.3, 0) coordinate (n1)
	(2.3, -2) coordinate (n2)
	($(n1)+(0.7,0)$) coordinate (n11)
	($(n2)+(0.7,0)$) coordinate (n22)
	
	%Nur zu Hilfszwecken
%	(n1) node[above] {$n1$}
%	(n2) node[above] {$n2$}
%	(n11) node[above] {$n11$}
%	(n22) node[above] {$n22$}
	
	
	(0, 0) to [american inductor, o-] (n1)
	(n1) to[C, *-*] (n2)
	(n1) -- (n11) to[american inductor] (n22) -- (n2)
	(n2) to[short, -o] (0, -2)
	;
\end{circuitikz}}
	\end{minipage}
	\begin{minipage}{0.65\textwidth}
		Es wird zwischen minimalen und nicht minimalen Eintoren unterschieden. Nicht minimale Eintore können auf ein minimales Eintor reduziert werden ohne das die Änderung von aussen bemerkt wird.
	\end{minipage}\\[5pt]
	Da die RET nur aus $L$ und $C$ bestehen hat das RET bei $\omega = 0$ und bei $\omega \rightarrow \infty$ jeweils eine Pol oder eine Nullstelle.
\end{karte}

\begin{karte}{Wie sieht der Reaktanzverlauf des folgenden RET aus und was für ein Typ hat das RET?\\[5pt]
	%Autor: Simon Walker
%Version: 1.0
%Datum: 15.04.2020
%Lizenz: CC BY-NC-SA

\begin{circuitikz}
	%\HelpCords{-2}{2}{5}{-5}
	
	\draw
	%Knoten
	(2.3, 0) coordinate (n1)
	(2.3, -2) coordinate (n2)
	($(n1)+(0.7,0)$) coordinate (n11)
	($(n2)+(0.7,0)$) coordinate (n22)
	
	%Nur zu Hilfszwecken
%	(n1) node[above] {$n1$}
%	(n2) node[above] {$n2$}
%	(n11) node[above] {$n11$}
%	(n22) node[above] {$n22$}
	
	
	(0, 0) to [american inductor, o-] (n1)
	(n1) to[C, *-*] (n2)
	(n1) -- (n11) to[american inductor] (n22) -- (n2)
	(n2) to[short, -o] (0, -2)
	;
\end{circuitikz}}
	\flushleft
	\begin{minipage}{0.55\textwidth}
		%Autor: Jürg Rast
%Datum: 04.04.2012
%Lizenz: CC BY-SA
%Grundversion von https://github.com/jrast/ELT4-Notizen/blob/master/tikzPictures/RET/ReaktanzLTyp.tex

%Geändert von: Simon Walker
%Datum: 15.04.2020
%Lizenz: CC BY-NC-SA


\usepgflibrary{shapes.misc}
\begin{tikzpicture}[smooth, xscale=0.5, yscale=0.5]
% Achsen
\draw[->, thick] (-0.2,0) -- +(8.4,0) node[right] {$\omega$}; % Horizontal
\draw[->, thick] (0,-2) -- +(0,4) node[above] {$X(\omega)$}; % Vertikal

% Plots
\draw[color=green!70!black, thick] plot[domain=0:2] (\x,{\x/8+tan(\x*3/4 r)/8}); % Erster Tan
%\draw[color=green!70!black, thick] plot[domain=2.01:4.23] (\x,{tan(\x r)}); % zweiter Tan
\draw[color=green!70!black, thick] plot[domain=2.4:8] (\x,{ (\x/3.2) -3.8/(\x -1.01) }); % letzte kurve

% Poolstellen
\draw[dashed, thick, draw=red] (2.1,-2.1) -- +(0,4.2); % Poolstelle 1
\node[cross out, draw=red, thick] (wr1) at (2.1,0) {};


\draw[dashed] plot[domain=0:8] (\x, {\x/4});
\node (wL) at (6.1,2.4) {$\omega \rightarrow \infty$};

% Nullstellen
\node[rounded rectangle, draw=blue, thick] at(0,0) {};
\node[rounded rectangle, draw=blue, thick] at(4,0) {};
\end{tikzpicture}
	\end{minipage}
	\begin{minipage}{0.4\textwidth}
		Bei diesem Beispiel handelt sich um ein L-Typ
	\end{minipage}
	Als erstes wird das Verhalten des RET bei DC und bei sehr grossen Frequenzen analysiert. Für DC leiten beide Induktivitäten also ist dort eine NS. Für hohe Frequenzen sperrt die Erste Spule also ist dort eine PS.\\
	Da es bei dieser Schaltung um ein minimales RET handelt und aus drei Bauteilen besteht, hat der Reaktanzverlauf drei Pol/Null stellen. Diese treten abwechselnd auf.
\end{karte}

\begin{karte}{Wie sieht der Reaktanzverlauf des folgenden RET aus?\\[5pt]
		%Autor: Simon Walker
%Version: 1.0
%Datum: 15.04.2020
%Lizenz: CC BY-NC-SA

\begin{circuitikz}
	%\HelpCords{-2}{2}{5}{-5}
	
	\draw
	%Knoten
	(3, 0) coordinate (n1)
	(3, -2) coordinate (n2)
	($(n1)+(0.7,0)$) coordinate (n11)
	($(n2)+(0.7,0)$) coordinate (n22)
	(0.7, 0) coordinate (n3)
	(0.7, -2) coordinate (n4)
	(0, 0) coordinate (n5)
	(0, -2) coordinate (n6)
	
	
	%Nur zu Hilfszwecken
%	(n1) node[above] {$n1$}
%	(n2) node[above] {$n2$}
%	(n11) node[above] {$n11$}
%	(n22) node[above] {$n22$}
%	(n3) node[above] {$n3$}
%	(n4) node[above] {$n4$}
%	(n5) node[above] {$n5$}
%	(n6) node[above] {$n6$}
	
	(n5) to[short,o-] (n3)
	(n3) to[C, *-*] (n4)
	(n3) to [american inductor] (n1)
	(n1) to[C, *-*] (n2)
	(n1) -- (n11) to[american inductor] (n22) -- (n2)
	(n2) to[short, -o] (n6)
	;
\end{circuitikz}}
	\centering
	%Autor: Jürg Rast
%Datum: 04.04.2012
%Lizenz: CC BY-SA
%Grundversion von https://github.com/jrast/ELT4-Notizen/blob/master/tikzPictures/RET/ReaktanzPTyp.tex

%Geändert von: Simon Walker
%Datum: 15.04.2020
%Lizenz: CC BY-NC-SA

\usepgflibrary{shapes.misc}
\begin{tikzpicture}[smooth, xscale=0.7, yscale=0.5]
% Achsen
\draw[->, thick] (-0.2,0) -- +(6.2,0) node[right] {$\omega$}; % Horizontal
\draw[->, thick] (0,-2) -- +(0,4) node[above] {$X(\omega)$}; % Vertikal

% Plots
\draw[color=green!70!black, thick] plot[domain=0.01:0.9] (\x, {tan( (\x *1.2) r )        }); % Erster Tan
\draw[color=green!70!black, thick] plot[domain=1.3:3.15] (\x,    {tan( (\x*1.2 -2.7) r)    }); % Erster Tan
\draw[color=green!70!black, thick] plot[domain=6.5:8.6] (\x-2.8,  { ((0.25 * \x) -1/(\x -6)) -2 }); % letzte kurve

% Poolstellen
\draw[dashed, thick, draw=red] (1,-2.1) -- +(0,4.2); % Poolstelle 1
\node[cross out, draw=red, thick] at (1,0) {};

\draw[dashed, thick, draw=red] (3.4,-2.1) -- +(0,4.2); % Poolstelle 2
\node[cross out, draw=red, thick] at (3.4,0) {};



% Nullstellen
\node[rounded rectangle, draw=blue, thick] at(0,0) {};
\node[rounded rectangle, draw=blue, thick] at(2.25,0) {};

\end{tikzpicture}\\
	\flushleft
	\vspace{-5pt}
	Als erstes wird das Verhalten des RET bei DC und bei sehr grossen Frequenzen analysiert. Für DC leiten beide Induktivitäten also ist dort eine NS. Für hohe Frequenzen leiten Beide Kapazitäten also ist dort ebenfalls NS.\\
	Da es bei dieser Schaltung um ein minimales RET handelt und aus vier Bauteilen besteht, hat der Reaktanzverlauf vier Pol/Null stellen. Diese treten abwechselnd auf.
\end{karte}

\begin{karte}{Ist folgendes RET ein minimales Eintor?\\[5pt]
		%Autor: Simon Walker
%Version: 1.0
%Datum: 15.04.2020
%Lizenz: CC BY-NC-SA

\begin{circuitikz}
	%\HelpCords{-2}{2}{5}{-5}
	
	\draw
	%Knoten
	(3, 0) coordinate (n1)
	(3, -2) coordinate (n2)
	($(n1)+(0.7,0)$) coordinate (n11)
	($(n2)+(0.7,0)$) coordinate (n22)
	(0.7, 0) coordinate (n3)
	(0.7, -2) coordinate (n4)
	(0, 0) coordinate (n5)
	(0, -2) coordinate (n6)
	
	
	%Nur zu Hilfszwecken
%	(n1) node[above] {$n1$}
%	(n2) node[above] {$n2$}
%	(n11) node[above] {$n11$}
%	(n22) node[above] {$n22$}
%	(n3) node[above] {$n3$}
%	(n4) node[above] {$n4$}
%	(n5) node[above] {$n5$}
%	(n6) node[above] {$n6$}
	
	(n5) to[short,o-] (n3)
	(n3) to[american inductor, *-*] (n4)
	(n3) to [american inductor] (n1)
	(n1) to[C, *-*] (n2)
	(n1) -- (n11) to[american inductor] (n22) -- (n2)
	(n2) to[short, -o] (n6)
	;
\end{circuitikz}}
	Nein! Denn es hat ein Ring aus Induktivitäten. In diesem Ring kann ein Strom fliesen. Dieser Strom ist von aussen nicht sichtbar. Eine Induktivität kann ersatzlos gestrichen werden, wenn dementsprechend die anderen Induktivitäten angepasst werden. Danach ist von aussen kein Unterschied feststellbar.\\
	\begin{center}
		%Autor: Simon Walker
%Version: 1.0
%Datum: 15.04.2020
%Lizenz: CC BY-NC-SA

\begin{circuitikz}
	%\HelpCords{-2}{2}{5}{-5}
	
	\draw
	%Knoten
	(3, 0) coordinate (n1)
	(3, -2) coordinate (n2)
	($(n1)+(0.7,0)$) coordinate (n11)
	($(n2)+(0.7,0)$) coordinate (n22)
	(0.7, 0) coordinate (n3)
	(0.7, -2) coordinate (n4)
	(0, 0) coordinate (n5)
	(0, -2) coordinate (n6)
	

	
	%Nur zu Hilfszwecken
%	(n1) node[above] {$n1$}
%	(n2) node[above] {$n2$}
%	(n11) node[above] {$n11$}
%	(n22) node[above] {$n22$}
%	(n3) node[above] {$n3$}
%	(n4) node[above] {$n4$}
%	(n5) node[above] {$n5$}
%	(n6) node[above] {$n6$}
	
	(n5) to[short,o-] (n3)
	(n3) to[american inductor, *-*] (n4)
	(n3) to [american inductor] (n1)
	(n1) to[C, *-*] (n2)
	(n1) -- (n11) to[american inductor] (n22) -- (n2)
	(n2) to[short, -o] (n6)
	;
	
	\draw[very thick, blue, rounded corners=5mm]
	(n3) -- (n11) -- (n22) -- (n4) -- cycle;
\end{circuitikz}
	\end{center}
\end{karte}

\begin{karte}{Welche Eigenschaften eines RET sind in der Praxis am unwahrscheinlichsten?}
	Es gibt zwei Probleme:
	\begin{compactitem}
		\item Impedanz wird nie 0 erreichen
		\item Die Impedanz kann nicht undendlich gross werden.
	\end{compactitem}
	Allerdings wird die Impedanz sich nicht komplett anders verhalten als die Reaktanz.
	%Autor: Simon Walker
%Version: 1.0
%Datum: 18.04.2020
%Lizenz: CC BY-NC-SA

\begin{tikzpicture}[smooth, xscale=0.7, yscale=0.7]
% Achsen
\draw[->, thick] (-0.2,0) -- +(5.2,0) node[right] {$\omega$}; % Horizontal
\draw[->, thick] (0,-2) -- +(0,4) node[above] {$\color{red}X \color{black}, \color{green} \left|Z\right|$}; % Vertikal

% Plots
\draw[color=red, thick] plot[domain=0.01:0.9] (\x*1.5,{tan((\x *1.2) r )}); % Erster Tan
\draw[color=red, thick] plot[domain=1.3:3.15] (\x*1.5,{tan((\x*1.2 -2.7) r)}); % Zweiter Tan
%Impedanz
%\draw[color=green, thick] plot[domain=0.01:0.9] (\x*1.5,{0.3+0.6*tan((\x *1.2) r )}); % Erster Tan

\draw [green, thick] plot [smooth] coordinates{
(0, 0.3) (0.5, 0.5) (1, 1) 
(1.5, 1.8)  (2, 1) (2.5, 0.5)
(3, 0.3) (3.5, 0.3) (4, 0.5) 
(4.5, 1) (5, 1.6) (5.5, 1.8)};


\draw [blue, thick] (3.34, 0) circle[radius=0.25];
\draw [blue, thick] (0, 0) circle[radius=0.25];
\draw [brown, thick] (1.35, 1.85) circle[radius=0.25];
\draw [brown, thick] (1.95, -2.15) circle[radius=0.25];
\draw [brown, thick] (4.73, 1.9) circle[radius=0.25];

\node [below] at (3.8, -0.2) {$\scriptstyle \color{blue} \left|Z\right| \ne 0 \Omega$};
\node at (3, 2) {$\scriptstyle \color{brown} \left|Z\right| \not\rightarrow \infty$};

\end{tikzpicture}

\end{karte}



	\kommentar{Leitungen}

\begin{karte}{Was ist eine Leitung?}
	Für uns ist eine Leitung eine Wellenführung.\\
	\divTwo{
		\textbf{Ungeführte Welle:}
		%Autor: Simon Walker
%Version: 1.0
%Datum: 24.04.2020
%Lizenz: CC BY-NC-SA

\begin{tikzpicture}[smooth, xscale=0.5, yscale=0.5]
	%\HelpCords{0}{0}{5}{4}
	
	\foreach \x in {0.6, 1.2, ...,4}{
		\draw[green] (\x, 0) arc (0:45:\x);	
	}
	
	\draw[<->, thick] (3, 3) -- (0, 0) -- (4.5, 0) node[right] {$\vec{r}$};
	\node[above right] at (3, 3) {$\vec{r}$};
\end{tikzpicture}
\\
		\scalebox{.7}{%Autor: Simon Walker
%Version: 1.0
%Datum: 24.04.2020
%Lizenz: CC BY-NC-SA

\begin{tikzpicture}[x={(0,0.8cm)}, y={(-0.55cm,-0.55cm)}, z={(1.2cm, -0.15cm)}]
		
	\def\cycles{1.75} %Anz Schwingungen
	\def\lenght{3}%länge in z richtung
	
	
	\draw [->, thick] (0,0,0)  -- (1.5,0,0) node[above] {\Large$x$};
	\draw [->, thick] (0,0,0) -- (0,1.2,0) node[below left] {\Large$y$};
	\draw [->, thick] (0,0,0) -- (0,0,\lenght+0.5) node[right] {\Large$z$};
	
	\draw[thick, green] %E-Feld Plot
	plot[domain=0:\lenght, samples=200] 
	({1.1*cos(deg(2*\cycles*pi*\x/\lenght))}, 0, \x);
	
	\draw[thick, blue] %M-Feld Plot
	plot[domain=0:\lenght, samples=200] 
	(0, {0.7*cos(deg(2*\cycles*pi*\x/\lenght))}, \x);
	
	\fill[opacity=.1,green!80] %E-Feld füllung
	(0,0,0) --
	plot[domain=0:\lenght, samples=200] 
	({1.1*cos(deg(2*\cycles*pi*\x/\lenght))}, 0, \x)  --
	(0,0,0);
	
	\fill[opacity=.1,blue!80] %M-Feld füllung
	(0,0,0) --
	plot[domain=0:\lenght, samples=200] 
	(0, {0.7*cos(deg(2*\cycles*pi*\x/\lenght))}, \x)  --
	(0,0,0);
	
	\node[green] at (1.3,0,0.3) {\Large$E$}; %Beschriftungen
	\node[blue] at (0,1.1,0.3) {\Large$B$};
	
\end{tikzpicture}
}
		\centering
		$\vec{k} \cdot \vec{r}$
		
	}{
		\textbf{Geführte Welle:}\\[5pt]
		%Autor: Simon Walker
%Version: 1.0
%Datum: 24.04.2020
%Lizenz: CC BY-NC-SA

\begin{tikzpicture}[smooth, xscale=0.4, yscale=0.4]
%	\HelpCords{0}{-1}{9}{1}

	\draw[thick] (-0.5, 0) -- (1, 0);
	\draw[thick] (8, 0) -- (9.5, 0);
	
	\draw[thick] (1, 0) ellipse circle [x radius=0.6, y radius=1];
	
	\draw[thick, fill=white] (8.2, 0) ellipse circle [x radius=0.6, y radius=1];
	\fill[white] (7,-1) rectangle (8.2,1);
	
	\foreach \x in {1.8, 2.4, ...,8}{
		\draw[green] (\x, -1) arc (-30:30:2);	
	}
	
	\draw[thick] (1, 1) -- (8.2, 1);
	\draw[thick] (1, -1) -- (8.2, -1);
	
	\foreach \x in {2.5, 4.25, ..., 8}{
		\draw[thick, blue] (\x, 0) ellipse circle [x radius=0.75, y radius=0.2];
	}

	\draw[thick, ->] (1, -1.8) -- (9, -1.8) node[right] {$z$};
\end{tikzpicture}

		\begin{center}
			$k\cdot z = \beta z \quad \Rightarrow \quad k=\beta$
		\end{center}
		Im Beispiel ist ein Koaxialleiter zu sehen. Das $B$-Feld ist Ortogonal zum $E$-Feld. Die gesamte Anordnung ist Symetrisch.
	}
\end{karte}

\begin{karte}{Wie Sieht das Schema des Leitungsmodells aus?}
	Das Leitungsmodell bezieht sich auf einen kurzen Leitungsausschnitt. Dadurch erhalten wir längenunabhängige Kenngrössen die sogenanten Leitungsbeläge.\\
	%\divTwo{
	%\begin{figure}[H]
	%	\centering
		\scalebox{.49}{%Autor: Simon Walker
%Version: 1.0
%Datum: 04.06.2020
%Lizenz: CC BY-NC-SA

\begin{circuitikz}
	%\HelpCords{0}{0}{8}{-4}
	
	
	\draw
	%Knoten
	(0, 0) coordinate (n1)
	
	($(n1)+(0.1,0)$) coordinate(n11)
	
	(1.5, 0) coordinate (n2)
	(3.5, 0) coordinate (n3)
	(5.5, 0) coordinate (n4)
	
	($(n4)+(0,-0.1)$) coordinate(n41)
	
	(8, 0) coordinate (n5)
	
	($(n5)+(-0.1,0)$) coordinate(n51)
	
	
	($(n4)+(0, -1)$) coordinate (n6)
	($(n6)+(0, +0.1)$) coordinate (n60)
	($(n6)+(-0.8, 0)$) coordinate (n61)
	($(n6)+(0.8, 0)$) coordinate (n62)
	
	($(n6)+(0, -2)$) coordinate (n7)
	($(n61)+(0, -2)$) coordinate (n71)
	($(n62)+(0, -2)$) coordinate (n72)
		
	($(n7)+(0, -0.5)$) coordinate (n9)
	
	(8, -3.5) coordinate (n8)
	(0, -3.5) coordinate (n10)
	
	
	
	%Nur zu Hilfszwecken
%	(n1) node[above] {$n1$}
%	(n2) node[above] {$n2$}
%	(n3) node[above] {$n3$}
%	(n4) node[above] {$n4$}
%	(n5) node[above] {$n5$}
%	(n6) node[above] {$n6$}
%	(n7) node[above] {$n7$}
%	(n8) node[above] {$n8$}
%	(n9) node[above] {$n9$}
%	(n10) node[above] {$n10$}
%	
%	(n11) node[below] {$n11$}
%	(n41) node[below] {$n41$}
%	(n51) node[below] {$n51$}
%	(n60) node[below] {$n60$}
%	(n61) node[above] {$n61$}
%	(n62) node[above] {$n62$}
%	(n71) node[above] {$n71$}
%	(n72) node[above] {$n72$}
	
	(n1) to [short,o-] (n11)
	(n11) to[short,color=red,i=${\color{red} i(z,t)}$] (n2)
	(n2) to[R=$R'dz$] (n3)
	(n3) to [american inductor=$L'dz$] (n4)
	(n4) to[short, color=red, i=${\color{red} i(z+dz,t)}$] (n51)
	(n51) to[short,-o] (n5)
	
	(n4) to[short, *-] (n41)
	(n41) to[short, color=red, i=${\color{red} \Delta i(z,t)}$] (n60)
	(n60) to[short, -*] (n6)
	(n6) to[short] (n61)
	(n6) to[short] (n62)
	(n61) to[R,l_=$G'dz$] (n71)
	(n62) to[C=$C'dz$] (n72)	
	(n7) to[short] (n71)
	(n7) to[short] (n72)
	(n7) to[short, *-*] (n9)
	
	(n10) to[short, o-o] (n8)
	;
	
	\draw[|<->|] ($(n10)+(0,-0.7)$) --
	node[above] {$dz$}
	($(n8)+(0,-0.7)$);
	
	\draw[blue,->] (0.3, 0.6) to[out=10,in=170] (7.3, 0.6);
	\node[above, blue] at (3.75, 0.9) {$\Delta u(z)$};
	
	
	\draw[blue,->] ($(n1)+(0, -0.3)$) --
	node[left, blue] {$u(z,t)$}
	($(n10)+(0, +0.3)$);
	
	\draw[blue,->] ($(n5)+(0, -0.3)$) --
	node[right, blue] {$u(z+dz,t)$}
	($(n8)+(0, +0.3)$);
	
\end{circuitikz}}\\[5pt]
	%\end{figure}
	%}{
	\tiny
	\renewcommand\arraystretch{1.7}
	\begin{tabular}{cl}
		$R'=\frac{dR}{dz}$ & Widerstandsbelag in $\Omega/m$\\
		$L'=\frac{dL}{dz}$ & Induktivitätsbelag in $H/m$\\
		$G'=\frac{dG}{dz}$ & Isolationsbelag in $S/m$\\
		$C'=\frac{dC}{dz}$ & Kapazitätsbelag in $F/m$\\
	\end{tabular}
	\normalsize
	%}


\end{karte}

\begin{karte}{Was ist eine TEM-Welle?}
	TEM steht für transversalelektomagnetisch. Eine TEM-Welle hat nur Komponenten, welche senkrecht zur Ausbreitungsrichtung sind. Im Bild ist $z$ die Ausbreitungsrichtung der Welle und steht senkrecht zu $\vec{E}$ und $\vec{H}$. \\[10pt]
	%Autor: Simon Walker
%Version: 1.0
%Datum: 28.04.2020
%Lizenz: CC BY-NC-SA

\begin{tikzpicture}[x={(0, 0.8cm)}, y={(-0.5cm,-0.5cm)}, z={(0.8cm, 0)}]

\def\l{7} %länge der z-Achse
\def\s{1.2} %Quadratgrösse (eine Seite ist s*2)
\def\p{5} %Positoon des Quadrates

\draw[domain=90:180,smooth,variable=\x] plot ({sin(\x)*0.5},{0}, {1.5+cos(\x)*0.5});
\draw[domain=0:90,smooth,variable=\x] plot ({0},{sin(\x)*0.5}, {1.5+cos(\x)*0.5});
\draw[fill=black] (0.2, 0, 1.3) circle[radius=0.03];
\draw[fill=black] (0, 0.2, 1.7) circle[radius=0.03];
\draw[very thick, green, ->] (0, 0, 1.5) -- (1, 0, 1.5) node[above, green] {$\vec{E}$};
\draw[very thick, blue, ->] (0, 0, 1.5) -- (0, 1, 1.5) node[below left, blue] {$\vec{H}$};

\draw[thick] (0, 0, 0) -- (0, 0, \p); %Beginn der Z-Achse

\draw[fill=gray!10] (-\s, -\s, \p) -- (-\s, \s, \p) -- (\s, \s, \p) -- (\s, -\s, \p) -- cycle;
\draw[->,domain=-75:195,smooth,variable=\x, blue, thick] plot ({cos(\x)*0.5},{sin(\x)*0.5}, {\p});
\node[blue] at (-0.6, 0.5, \p) {$B(I)$};
\draw[->, green, thick] (0, 0, \p) -- (1, 0, \p) node[right, green] {$U$};
\draw[thick, ->] (0, 0, \p) -- (0, 0, \l) node[right] {$z$}; %Ende der Z-Achse

\end{tikzpicture}

\end{karte}

\begin{karte}{Was bedeutet der Begriff Wellenimpedanz?}
	Die Wellenimpedanz ist das Verhältnis zwischen Spannung und Strom der Welle. Achtung: Dies ist nicht gleich der gemessenen Spannung und Strom an einem gewissen Punkt $z$ in der Leitung!\\[5pt]
	%Autor: Simon Walker
%Version: 1.0
%Datum: 30.04.2020
%Lizenz: CC BY-NC-SA

\begin{tikzpicture}[smooth, xscale=0.6, yscale=0.6]
%	\HelpCords{0}{-1}{9}{1}

	\draw[thick] (-1, 0) -- (1, 0);
	\draw[thick] (8, 0) -- (10.5, 0);
	
	\draw[thick] (1, 0) ellipse circle [x radius=0.6, y radius=1];
	
	\draw[thick, fill=white] (8.2, 0) ellipse circle [x radius=0.6, y radius=1];
	\fill[white] (7,-1) rectangle (8.2,1);
	
	
	\draw[thick] (1, 1) -- (8.2, 1);
	\draw[thick] (1, -1) -- (8.2, -1);
	
	%Pfeile
	\draw [->, blue] (2, 0) -- ++(0.3,0) 
	arc [radius=0.2, start angle=180, end angle = 0] 
	arc [radius=0.2, start angle=-180, end angle = 0]
	arc [radius=0.2, start angle=180, end angle = 0] -- ++(0.3,0);
	\node [blue] at (3, 0.6) {$U_+$};
	\node [blue] at (3, -0.6) {$I_+$};
	
	\draw [->, blue] (8, 0) -- ++(-0.3,0) 
	arc [radius=0.2, start angle=0, end angle = 180] 
	arc [radius=0.2, start angle=0, end angle = -180]
	arc [radius=0.2, start angle=0, end angle = 180] -- ++(-0.3,0);
	\node [blue] at (7, 0.6) {$U_-$};
	\node [blue] at (7, -0.6) {$I_-$};

	\draw[thick, ->] (1, 1.7) -- (9, 1.7) node[right] {$z$};
	
	\draw[very thick, ->] (2, -1.3) -- ++(0, -1.2) -- ++(1.2, 0) node [right]
	{$Z_0 = \color{blue} \dfrac{U_+}{I_+} \color{black} = 
	\color{blue} \dfrac{U_-}{I_-} \color{red} \ne \dfrac{U(z)}{I(z)}$}; 
	\node[right] at (3.2, {-2.5-1.3}) 
	{$\color{white} Z_0 \color{black} = \sqrt{\frac{R'+j \omega L'}{G' +j \omega C'}}= Z(\omega) \text{\tiny{\quad gemäss Leitungsmodel}}$} ;
	
\end{tikzpicture}
 
\end{karte}

\begin{karte}{Was bezeichnet man als Ausbreitungsgeschwindigkeit?}
	Für uns ist meistens mit der Ausbreitungsgeschwindigkeit die Phasengeschwindigkeit gemeint.
	\begin{equation*}
		v = \frac{dz}{dt} = v_{ph} 
	\end{equation*}
	Am einfachsten wird geschaut wie sich der Peak der Welle fortbewegt.
	\vspace{-10pt}
	\begin{figure}[h]
		\centering
		%Autor: Simon Walker
%Version: 1.0
%Datum: 30.04.2020
%Lizenz: CC BY-NC-SA

\begin{tikzpicture}[xscale=0.7, yscale=0.7]
	%\HelpCords{0}{-1}{9}{1}
	
	\def\l{8} %Länge der Schwingungen
	\def\o{0.9} %Offset (Positiv bedeutet Rot links von blau)
	\def\f{0.3} %Frequenz der Schwingunen
	
	\tikzmath{
		\p1 = 450/(360*\f); %Peek1
		\p2 = \p1 - \o; %Peek2
		\pl = (\p1+\p2)/2;
	}
	
	\draw[blue] plot[domain=0:\l, samples=200] 
	(\x, {sin(360*\f*\x});
	\draw[red] plot[domain=0+\o:\l+\o, samples=200] 
	(\x-\o, {sin(360*\f*\x)});
	
	%\draw[green, *<-*] (\p2, 1) -- (\p1, 1);
	\draw [thick] (\p1, 1) -- (\p1, 1.5);
	\draw [thick](\p2, 1) -- (\p2, 1.5);
	\draw [->, thick] (\p1, 1.25) -- (\p2, 1.25);
	\node[above] at (\pl, 1.25) {$dz$};
	
	\draw[fill=green, green] (\p1, 1) circle [radius=0.08];
	\draw[fill=green, green] (\p2, 1) circle [radius=0.08];
	
\end{tikzpicture}

	\end{figure}
\end{karte}

\begin{karte}{Was ist die Ausbreitungskonstante?}
	Die Ausbreitungskonstante $\gamma$ ist eine komplexe Grösse und hat die Einheit $Np/m + jrad/m$. $Np/m$ ist (Neper/Meter).\\[7pt]
	\begin{minipage}[c]{0.38\textwidth}
		\begin{equation*}
		\gamma = \alpha + j \beta
		\end{equation*}
	\end{minipage}
	\begin{minipage}[t]{0.68\textwidth}
		\begin{tabular}{ccl}
			$\gamma$ & => & Ausbreitungskonstante\\
			$\alpha$ & => & Dämpfungskonstante\\
			$\beta$  & => & Phasenkonstante
		\end{tabular}
	\end{minipage}\\[7pt]
	
	Eine Welle kann mittels der Ausbreitungskonstante definiert werden:
	\begin{equation*}
		U_+ e^{-\gamma z}
	\end{equation*}
	Wobei $U_+ e^{-\gamma z}$ eine Welle in die Positive $z$ Richtung ist mit der komplexen Amplitude $U_+=|U_+| e^{j \varphi_+}$
\end{karte}

\begin{karte}{Wie entstehen durch Leitungen Verzerrungen und wie können diese verhindert werden?}
	Eine Möglichkeit ist die \textbf{Dispersion}. Dies ist dann der Fall wenn $\beta(\omega)$, $\gamma(\omega)$, $Z(\omega)$, $v(\omega)$ Frequenzabhänig sind.\\
	Eine andere Möglichkeit für Verzerrungen sind Reflexionen. Die Reflexionen treten immer Zeitverzögert auf ($\tau$).\\
	\scalebox{.8}{%Autor: Simon Walker
%Version: 1.0
%Datum: 15.04.2020
%Lizenz: CC BY-NC-SA

\begin{circuitikz}
	%\HelpCords{0}{0}{8}{3}
	
	
	\draw
	%Knoten
	(0, 0) coordinate (n1)
	(0, 3) coordinate (n2)
	(2, 3) coordinate (n3)
	(6, 3) coordinate (n4)
	(7, 3) coordinate (n5)
	(7, 0) coordinate (n6)	
	
	
	%Nur zu Hilfszwecken
%	(n1) node[above] {$n1$}
%	(n2) node[above] {$n2$}
%	(n3) node[above] {$n3$}
%	(n4) node[above] {$n4$}
%	(n5) node[above] {$n5$}
%	(n6) node[above] {$n6$}
	
	(n2) to[sV] (n1) node[rground]{}
	(n2) to[R] (n3)
	(n4) to[short] (n5) to[R=${Z_a}$] (n6) node[rground]{}
	;
	
	%Leitung
	\draw
	(n3) to[short] ($(n3)+(0.5, 0)$)
	(n4) to[short] ($(n4)+(-2, 0)$);
	
	\draw[fill=white] ($(n4)+(-0.5, 0)$) ellipse circle [x radius=0.3, y radius=0.5];
	\fill[white] ($(n3)+(0.5, -0.5)$) rectangle ($(n4)+(-0.5, 0.5)$);
	\draw ($(n3)+(0.5, 0)$) ellipse circle [x radius=0.3, y radius=0.5];
	\draw ($(n3)+(0.5, -0.5)$) -- ($(n4)+(-0.5, -0.5)$);
	\draw ($(n3)+(0.5, +0.5)$) -- ($(n4)+(-0.5, +0.5)$);
	
	\draw[red,|->] ($(n3)+(0,0.8)$) -- 
	node[above, red] {$t=\tau$}
	($(n4)+(0,0.8)$);
	
	\path ($(n3)+(0,-0.5)$) -- 
	node[below] {$Z_0$}
	($(n4)+(0,-0.5)$);
	
	\draw[dashed] ($(n4)+(0,1)$) -- ($(n4)+(0,-3.5)$);
	
	\draw[<-, very thick] ($(n4)+(0,-1.5)$) -| ++(-0.8,-0.5) node[below] {$\Gamma$};
	

\end{circuitikz}}
\end{karte}

	\kommentar{Zweitore}

\begin{karte}{Worum handelt sich bei einem Zweitor?}
	Ein Zweitor ist ein Vierpol bei welchem immer zwei als Paar auftreten. Die Spannungen und Ströme sind wie im Schema eingezeichnet definiert. Es muss immer auf das Vorzeichen geachtet werden. Ansonsten funktionieren die gesamten Rechenregeln nicht.
	
	\centering{\scalebox{.9}{%Autor: Simon Walker
%Version: 1.0
%Datum: 25.06.2020
%Lizenz: CC BY-NC-SA


\begin{tikzpicture}

%Symbol
\draw[thick] (-1, -1.3) rectangle (1, 1.3);
\draw[thick] (-1,0.8) -- (-2,0.8);
\draw[thick, fill=white] (-2,0.8) circle (0.1);
\draw[thick] (-1,-0.8) -- (-2,-0.8);
\draw[thick, fill=white] (-2,-0.8) circle (0.1);

\draw[thick] (1,0.8) -- (2,0.8);
\draw[thick, fill=white] (2,0.8) circle (0.1);
\draw[thick] (1,-0.8) -- (2,-0.8);
\draw[thick, fill=white] (2,-0.8) circle (0.1);

%Strompfeile
\draw[red, thick] (-1.5, 0.8) node[above, red] {$i_1$} ++(-0.15, 0.15) -- ++(0.15, -0.15) -- ++(-0.15, -0.15);
\draw[red, thick] (-1.5, -0.8) node[above, red] {$i_1$} ++(0.15, 0.15) -- ++(-0.15, -0.15) -- ++(0.15, -0.15);

\draw[red, thick] (1.5, 0.8) node[above, red] {$i_2$} ++(0.15, 0.15) -- ++(-0.15, -0.15) -- ++(0.15, -0.15);
\draw[red, thick] (1.5, -0.8) node[above, red] {$i_2$} ++(-0.15, 0.15) -- ++(0.15, -0.15) -- ++(-0.15, -0.15);

%Spannungspfeile
\draw[blue, ->, thick, out=-110, in=110] (-2.1, 0.7) to (-2.1, -0.7);
\node[left, blue] at (-2.3,0){$u_1$};

\draw[blue, ->, thick, out=-60, in=60] (2.1, 0.7) to (2.1, -0.7);
\node[right, blue] at (2.3,0){$u_2$};


\end{tikzpicture}
}}	
\end{karte}

\begin{karte}{Was versteht man im Bezug auf Zweitore unter Passivität, Reziprozität und Symmetrie?}
	\textbf{Passivität:} Die Leistung die das Zweitor aufnimmt ist grösser als 0 ($P\ge 0$)\\
	\textbf{Reziprozität:} Ist die Symmetrie bezüglich dem Ausgang. Die Ströme durch die Amperemeter (blau umkreist) müssen gleich sein.\\
	\textbf{Symmetrie:} Die Ströme durch die Quelle (grün umkreist) müssen ebenfalls gleich sein.
	
	\centering{\scalebox{.7}{%Autor: Simon Walker
%Version: 1.0
%Datum: 25.06.2020
%Lizenz: CC BY-NC-SA


\begin{circuitikz}
%Symbol1
\draw[thick] (-1, -0.8) rectangle (1, 0.8);
\draw (-1, 0.5) to[short, -o] (-2, 0.5);
%\draw[thick] (-1,0.5) -- (-2,0.5);
%\draw[thick, fill=white] (-2,0.5) circle (0.1);
\draw (-1, -0.5) to[short, -o] (-2, -0.5);
%\draw[thick] (-1,-0.5) -- (-2,-0.5);
%\draw[thick, fill=white] (-2,-0.5) circle (0.1);

\draw[thick] (1,0.5) -- (2,0.5);
\draw[thick, fill=white] (2,0.5) circle (0.1);
\draw[thick] (1,-0.5) -- (2,-0.5);
\draw[thick, fill=white] (2,-0.5) circle (0.1);

%Strompfeile
\draw[red, thick] (-1.5, 0.5) node[above, red] {$i_1$} ++(-0.15, 0.15) -- ++(0.15, -0.15) -- ++(-0.15, -0.15);

\draw[red, thick] (1.5, 0.5) node[above, red] {$i_2$} ++(0.15, 0.15) -- ++(-0.15, -0.15) -- ++(0.15, -0.15);

%Quelle
\draw (-2,0.5) to[short] (-2,1) to[short] (-2.5, 1)	to[american voltage source] (-2.5 ,-1) to[short] (-2,-1) to[short] (-2, -0.5);

%Amperemeter
\draw (2,0.5) to[short] (2,1) to[short] (2.5, 1) to node[draw,circle,fill=white] {A} (2.5 ,-1) to[short] (2,-1) to[short] (2, -0.5);

%Einkreisen
\draw[green, very thick] (-1.5, 0.5) circle (0.5);
\draw[blue, very thick] (1.5, 0.5) circle (0.5);


\begin{scope}[shift={(7,0)}]
	%Symbol2
	\draw[thick] (-1, -0.8) rectangle (1, 0.8);
	\draw (-1, 0.5) to[short, -o] (-2, 0.5);
	%\draw[thick] (-1,0.5) -- (-2,0.5);
	%\draw[thick, fill=white] (-2,0.5) circle (0.1);
	\draw (-1, -0.5) to[short, -o] (-2, -0.5);
	%\draw[thick] (-1,-0.5) -- (-2,-0.5);
	%\draw[thick, fill=white] (-2,-0.5) circle (0.1);
	
	\draw[thick] (1,0.5) -- (2,0.5);
	\draw[thick, fill=white] (2,0.5) circle (0.1);
	\draw[thick] (1,-0.5) -- (2,-0.5);
	\draw[thick, fill=white] (2,-0.5) circle (0.1);
	
	%Strompfeile
	\draw[red, thick] (-1.5, 0.5) node[above, red] {$i_1$} ++(-0.15, 0.15) -- ++(0.15, -0.15) -- ++(-0.15, -0.15);
	
	\draw[red, thick] (1.5, 0.5) node[above, red] {$i_2$} ++(0.15, 0.15) -- ++(-0.15, -0.15) -- ++(0.15, -0.15);
	
	%Quelle
	\draw (-2,0.5) to[short] (-2,1) to[short] (-2.5, 1)	to node[draw,circle,fill=white] {A} (-2.5 ,-1) to[short] (-2,-1) to[short] (-2, -0.5);
	
	%Amperemeter
	\draw (2,0.5) to[short] (2,1) to[short] (2.5, 1) to[american voltage source] (2.5 ,-1) to[short] (2,-1) to[short] (2, -0.5);
	
	%Einkreisen
	\draw[green, very thick] (1.5, 0.5) circle (0.5);
	\draw[blue, very thick] (-1.5, 0.5) circle (0.5);
	
\end{scope}


\end{circuitikz}
}}
\end{karte}

	\kommentar{EMV}

	
	

	
\end{document}
