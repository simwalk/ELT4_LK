\kommentar{Recap ELT3}

\begin{karte}{Was ist eine Reaktanz}
	Die Reaktanz ($X$) ist der Blindwiderstand. Somit ist er der Imaginärteil der Impedanz ($Z$). 
	Die Reaktanz lässt sich folgendermassen berechnen:\\[10pt]
	\begin{minipage}[t]{0.48\textwidth}
		Für Kapazitäten:\\
		$X = - \dfrac{1}{\omega C}$
	\end{minipage}
	\begin{minipage}[t]{0.48\textwidth}
		Für Induktivitäten:\\
		$X = \omega L$
	\end{minipage}\\[10pt]
	Da die Impedanz $Z = jX$ ist folgt:\\[10pt]
	\begin{minipage}[t]{0.48\textwidth}
		Für Kapazitäten:\\
		$Z = jX = - \dfrac{j}{\omega C} = \dfrac{1}{j \omega C}$
	\end{minipage}
	\begin{minipage}[t]{0.48\textwidth}
		Für Induktivitäten:\\
		$Z = jX = j \omega L$
	\end{minipage}
\end{karte}
