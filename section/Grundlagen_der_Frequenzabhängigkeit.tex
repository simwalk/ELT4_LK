\kommentar{Frequenzabhängigkeit}

\begin{karte}{Was ist die Güte (einer Reaktanz) und wie ist sie definiert?}
	Die Güte eines reaktiven Elements ist definiert als das Verhältnis seiner Reaktanz (Imaginärteil der Impedanz) und seiner Resistanz (Realteil der Impedanz):\\
	\begin{equation*}
		Q=\frac{|\operatorname{Im} Z|}{\operatorname{Re} Z}=\frac{|X|}{R}
	\end{equation*}
	Grundsätzlich ist ein Bauteil besser je grösser die Güte ist.
\end{karte}

\begin{karte}{Was ist der Verlustfaktor $d$ und was sagt der Winkel $\delta$ aus?}
	Der Verlustfaktor ist der Kehrwert der Güte.
	\begin{equation*}
	d = \frac{1}{Q} = \frac{\operatorname{Re}}{|\operatorname{Im} Z|} = \tan (\delta)
	\end{equation*}
	Der Verlustwinkel $\delta$ wird mehrheitlich dafür eingesetzt, den Anteil der Wirkleistung elektrisch reaktiver Bauteile zu beschreiben und ist wie folgt definiert:
	\begin{equation*}
	\tan(\delta)=\frac{P}{\omega W}=\frac{\operatorname{Re} Z}{|\operatorname{Im} Z|}=\frac{R}{|X|}
	\qquad \delta = \frac{\pi}{2} - |\varphi|
	\end{equation*}
\end{karte}

\begin{karte}{Was ist die Güte einer Induktivität und einer Kapazität?}
	Für Induktivitäten:
	\begin{equation*}
		Q_L = \frac{X_L}{R_L} = \frac{\omega_L}{R_L} = \omega \tau_L
	\end{equation*}
	Für Kapazitäten:
	\begin{equation*}
		Q_C = \frac{|X_C|}{R_C} = \frac{1}{\omega R_C C} = \frac{1}{\omega \tau_C}
	\end{equation*}
	%TODO mit Theorie ergänzen ev. eine Grafik
\end{karte}