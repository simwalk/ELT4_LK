\kommentar{Leitungen}

\begin{karte}{Komplettiere folgendes Schema wenn gilt $\Gamma = 0.6$\\($U_{1-}$, $I_{1-}$, $P_{1-}$, $U_{2+}$, $I_{2+}$, $P_{2+}$):\\[2pt]
	\scalebox{0.62}{%Autor: Simon Walker
%Version: 1.0
%Datum: 16.07.2020
%Lizenz: CC BY-NC-SA

\begin{tikzpicture}[xscale = 0.8, yscale = 0.8]
	
	\coordinate (a1) at (0, 5);
	\coordinate (a2) at (1, 5);
	\coordinate (a3) at (4, 5);
	\coordinate (a4) at (6, 5);
	\coordinate (a5) at (8, 5);
	\coordinate (a6) at (11, 5);
	\coordinate (a7) at (12, 5);
	
	\coordinate (b1) at (0, 1);
	\coordinate (b2) at (1, 1);
	\coordinate (b3) at (4, 1);
	\coordinate (b4) at (6, 1);
	\coordinate (b5) at (8, 1);
	\coordinate (b6) at (11, 1);
	\coordinate (b7) at (12, 1);
	
	%Linien
	\draw[thick] (a1) -- (a2) (a3) -- (a5) (a6) -- (a7);
	\draw[thick] (b1) -- (b2) (b3) -- (b5) (b6) -- (b7);
	
	%Anschlüsse
	\draw[fill=white] (a1) circle (0.07);
	\draw[fill=white] (a4) circle (0.07);
	\draw[fill=white] (a7) circle (0.07);
	\draw[fill=white] (b1) circle (0.07);
	\draw[fill=white] (b4) circle (0.07);
	\draw[fill=white] (b7) circle (0.07);
	
	%Leitungen
	\draw[thick] ($(a2)-(0, 0.25)$) rectangle ($(a3)+(0, 0.25)$);
	\draw[thick] ($(a5)-(0, 0.25)$) rectangle ($(a6)+(0, 0.25)$);
	\draw[thick] ($(b2)-(0, 0.25)$) rectangle ($(b3)+(0, 0.25)$);
	\draw[thick] ($(b5)-(0, 0.25)$) rectangle ($(b6)+(0, 0.25)$);
	
	%gestrichelte Linie
	\draw[dashed, thick] ($(a4)+(0, 1)$) -- ($(b4)-(0, 1)$);
	
	%Wellen Pfeile (Spannung)
	\draw [blue] ($(a2)!0.5!(a3)$)++ (-0.8, 0.5) -- ++(0.3,0)
	coordinate (p);
	\begin{scope}[shift={(p)}]
	\draw[blue, ->] plot[domain=0:540] ({1/540*\x}, {sin(\x)/8}) -- ++(0.3, 0);
	\end{scope}
	\node[blue] at ($(a2)!0.5!(a3)+(0,1)$) {$U_{1+} = 1V$};
	 
	\draw [blue] ($(a5)!0.5!(a6)$)++ (-0.8, 0.5) -- ++(0.3,0) coordinate (p);
	\begin{scope}[shift={(p)}]
	\draw[blue, ->] plot[domain=0:540] ({1/540*\x}, {sin(\x)/8}) -- ++(0.3, 0);
	\end{scope}
	\node[blue] at ($(a5)!0.5!(a6)+(0,1)$) {$U_{2+}= ?$};
	 
	\draw [blue, <-] ($(b2)!0.5!(b3)$)++ (-0.8, -0.5) -- ++(0.3,0) coordinate (p);
	\begin{scope}[shift={(p)}]
	\draw[blue] plot[domain=0:540] ({1/540*\x}, {sin(\x)/8}) -- ++(0.3, 0);
	\end{scope}
	\node[blue] at ($(b2)!0.5!(b3)-(0,1)$) {$U_{1-}= ?$};
	 
	 %Wellen Pfeile (Strom)
	 \draw [red] ($(a2)!0.5!(a3)$)++ (-0.8, -0.5) -- ++(0.3,0) coordinate (p);
	\begin{scope}[shift={(p)}]
	\draw[red, ->] plot[domain=0:540] ({1/540*\x}, {sin(\x)/8}) -- ++(0.3, 0);
	\end{scope}
	\node[red] at ($(a2)!0.5!(a3)-(0,1)$) {$I_{1+} = 1A$};
	 
	\draw [red] ($(a5)!0.5!(a6)$)++ (-0.8, -0.5) -- ++(0.3,0) coordinate (p);
	\begin{scope}[shift={(p)}]
	\draw[red, ->] plot[domain=0:540] ({1/540*\x}, {sin(\x)/8}) -- ++(0.3, 0);
	\end{scope}
	\node[red] at ($(a5)!0.5!(a6)-(0,1)$) {$I_{2+}= ?$};
	 
	\draw [red] ($(b2)!0.5!(b3)$)++ (-0.8, +0.5) -- ++(0.3,0) coordinate (p);
	\begin{scope}[shift={(p)}]
	\draw[red, ->] plot[domain=0:540] ({1/540*\x}, {sin(\x)/8})-- ++(0.3, 0);
	\end{scope}
	\node[red] at ($(b2)!0.5!(b3)+(0,1)$) {$I_{1-}= ?$};
	 
	%Leistungspfeile
	\draw[very thick, ->, brown] ($(a1)!0.5!(b1)+(0,-0.5)$) -- node[above] {$P_{1+} = 1W$} ++(1.5,0);
	
	\draw[very thick, ->, brown] ($(a4)!0.5!(b4)+(-0.4,-0.5)$) -- node[above] {$P_{1-} = ?$} ++(-1.5,0);
	\draw[very thick, ->, brown] ($(a4)!0.5!(b4)+(0.4,-0.5)$) -- node[above] {$P_{2+}= ?$} ++(1.5,0);
	
\end{tikzpicture}}}
	\begin{align*}
		U_{1-} &= \Gamma \cdot U_{1+} = \mathbf{0.6V}\\
		I_{1-} &= \Gamma \cdot I_{1+} = \mathbf{0.6A}\\
		P_{1-} &= U_{1-} \cdot I_{1-} = \Gamma^2 \cdot P_{1+} = \mathbf{0.36W}\\
		&\\
		U_{2+} &= (1+\Gamma) \cdot U_{1+} = \mathbf{1.6V}\\
		I_{2+} &= (1-\Gamma) \cdot I_{1+} = \mathbf{0.4A}\\
		P_{2+} &= (1+\Gamma) \cdot (1-\Gamma) = (1 - \Gamma^2) \cdot P_{1+} = \mathbf{0.64W}
	\end{align*}
\end{karte}

\begin{karte}{Wo ist die Impedanz/Admittanz im Smithchart $0$ und wo sind sie $\infty$.\\
	\scalebox{0.45}{%Autor: Simon Walker
%Version: 1.0
%Datum: 24.07.2020
%Lizenz: CC BY-NC-SA

\begin{tikzpicture}

\begin{smithchart}[width=10cm,
smithchart mirrored,
grid style={black!40!white},
yticklabels={,,},
xticklabels={,,}
]

\end{smithchart}
\begin{smithchart}[width=10cm,
grid style={black},
yticklabels={,,},
xticklabels={,,}
]


\end{smithchart}

\end{tikzpicture}
}}
	Die Nullstelle der Impedanz ist zugleich die $\infty$ stelle der Admittanz und befindet sich auf der linken Seite (blau). Die Nullstelle der Admittanz ist somit auf der rechten Seite zusammen mit der $\infty$ stelle der Impedanz (rot).
	\begin{tightcenter}
		\scalebox{0.40}{%Autor: Simon Walker
%Version: 1.0
%Datum: 24.07.2020
%Lizenz: CC BY-NC-SA

\begin{tikzpicture}

\begin{smithchart}[width=10cm,
smithchart mirrored,
grid style={black!40!white},
yticklabels={,,},
xticklabels={,,}
]
\end{smithchart}

\begin{smithchart}[width=10cm,
grid style={black},
yticklabels={,,},
xticklabels={,,}
]
\draw[blue,fill=blue] (0,0) circle (0.2cm);
\draw[red,fill=red] (900,0) circle (0.2cm);

\end{smithchart}

\node[above left] at (0,4.2cm) {\LARGE$Z=0$};
\node[below left] at (0,4.2cm) {\LARGE$Y=\infty$};
\node[above right] at (8.4,4.2cm) {\LARGE$Z=\infty$};
\node[below right] at (8.4,4.2cm) {\LARGE$Y=0$};

\end{tikzpicture}
}
	\end{tightcenter}
\end{karte}
