\kommentar{EMV}

\begin{karte}{Wie lassen sich kapazitiv eingekoppelte Störungen reduzieren?}
	Es gibt fünf Ansätze:\\
	$\bullet$ \textbf{Koppelkapazitäten minimieren:}
	\begin{compactitem}
		\item kurze Verbindungsleitungen
		\item grosser Abstand zwischen den betreffenden Leitungen
		\item Vermeidung von parallel geführten Leitungen
	\end{compactitem}
	$\bullet$ \textbf{Niederohmige Speisung:}
	\begin{compactitem}
		\item Signalspannungsquelle mit möglichst geringem $R_i$
		\item Verringern von $du/dt$ der Störspannung
	\end{compactitem}
	$\bullet$ \textbf{Abschirmung}\\
	$\bullet$ \textbf{Differentielle Signalübertragung}\\
	$\bullet$ \textbf{Verringerung der Flankensteilheiten}
\end{karte}

\begin{karte}{Wie lassen sich induktiv eingekoppelte Störungen reduzieren?}
	\begin{compactitem}
		\item Verringerung der Gegeninduktivität $M$ (maximieren des Abstands, minimieren der Schleifengrössen)
		\item Verdrillen von Hin- und Rückleitern
		\item Symmetrische (differentielle) Signalübertragung
		\item Reduzierung der Änderungsgeschwindigkeit des Störstromes
		\item Herabsetzen der Flussänderung $d \Phi / d t$ durch eine Kurzschlussschleife in unmittelbarer Nähe des gefährdeten Nutzkreises
		\item Abschirmen der Leitungen, Stromkreise und Baugruppen durch ferro-/ferrimagnetische Schirme
	\end{compactitem}
\end{karte}

\begin{karte}{Wie lassen sich galvanisch eingekoppelte und strahlungseingekoppelte Störungen reduzieren?}
	\textbf{Galvanische Kopplung:}
	\begin{itemize}
		\item Entkopplungskapazitäten
		\item getrennte Zuleitungen
	\end{itemize}
	\vspace{5pt}
	\textbf{Strahlungseinkopplung:}
	\begin{itemize}
		\item Schirmung (nicht beliebig dünn; Skin-Tiefe muss berücksichtigt werden)
	\end{itemize}
\end{karte}
