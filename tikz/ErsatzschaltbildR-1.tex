%Autor: Simon Walker
%Version: 1.0
%Datum: 19.04.2020
%Lizenz: CC BY-NC-SA

\begin{circuitikz}
%	\HelpCords{-1}{-3}{5}{1}
	
	\draw
	%Knoten
	(0, -2) coordinate (n11)
	(1, -2) coordinate (n12)
	(1, -2.75) coordinate (n13)
	(1, -1.25) coordinate (n14)
	(3, -2.75) coordinate (n15)
	(3, -1.25) coordinate (n16)
	(3, -2) coordinate (n17)
	(4, -2) coordinate (n18)
	
	(0, 0) coordinate (n21)
	(2, 0) coordinate (n22)
	(4, 0) coordinate (n23)
	
	
	%Nur zu Hilfszwecken
%	(n11) node[above] {$n11$}
%	(n12) node[above] {$n12$}
%	(n13) node[above] {$n13$}
%	(n14) node[above] {$n14$}
%	(n15) node[above] {$n15$}
%	(n16) node[above] {$n16$}
%	(n17) node[above] {$n17$}
%	(n18) node[above] {$n18$}
	
%	(n21) node[above] {$n21$}
%	(n22) node[above] {$n22$}
%	(n23) node[above] {$n23$}
	;
	%Serie L
	\draw[red]
	(n21) to[R=$R$, o-, color=red] (n22) 
	to [american inductor=$L$, -o, color=red] (n23)
	;
	
	
	%Parallel C
	\draw[blue]
	(n11) to[short, o-*, color=blue] (n12)
	(n13) to[short, color=blue] (n14)
	(n14) to[R=$R$, color=blue] (n16)
	(n13) to[C=$C$, color=blue] (n15)
	(n16) to[short, color=blue] (n15)
	(n17) to[short, *-o, color=blue] (n18)
	;
\end{circuitikz}
